
\section{Conclusion}
\label{sec:conclusion}

We have provided a rate-optimal community estimation algorithm for the homogeneous weighted stochastic block model. In the setting where the average degree is of order $\log n$ and the edge weight densities $p(x)$ and $q(x)$ are fixed, we have also characterized the exact recovery threshold. Our algorithm includes a preprocessing step consisting of transforming and discretizing the (possibly) continuous edge weights to obtain a simpler graph with edge weights supported on a finite discrete set. This approach may be useful for other network data analysis problems involving continuous distributions, where discrete versions of the problem are simpler to analyze.

Our paper is a first step toward understanding the weighted SBM under the same mathematical framework that has been so fruitful for the unweighted SBM. It is far from comprehensive, however, and many open questions remain. We describe a few here:
\begin{enumerate}
\item An important problem is to extend our analysis to the case of a \emph{heterogenous} stochastic block model, where edge weight distributions depend on the exact community assignments of both endpoints. In such a setting, Abbe and Sandon~\cite{AbbSan15} and Yun and Proutiere~\cite{yun2016optimal} have shown that a generalized information divergence---the CH divergence---governs the intrinsic difficulty of community recovery. We believe that a similar discretization-based approach should lead to analogous results in the case of a heterogeneous weighted SBM. The key challenge would be to show that discretization does not lose much information with respect to the CH-divergence.
\item Real-world networks often have nodes with very high degrees, which may adversely affect the accuracy of recovery methods for the stochastic block model. To solve this problem, degree-corrected SBMs \cite{zhao2012consistency, gao2016community} have been proposed as an effective alternative to regular SBMs. It is straightforward to extend the concept of degree-correction to the weighted SBM, but it is unclear whether our discretization-based approach would be effective in obtaining optimal error rates.
\item It is easy to extend our results to the weighted \emph{and} labeled SBMs if the number of labels is finite or assumed to be slowly growing. However, this excludes some interesting cases, including the setting where edge labels represent counts  from a Poisson distribution. We suspect that in such a situation, it may be possible to combine low-probability labels in a clever way to obtain a discretization that is again amenable to our approach. 
\end{enumerate}

%%% Local Variables:
%%% mode: latex
%%% TeX-master: "paper"
%%% End:
