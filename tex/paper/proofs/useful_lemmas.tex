\section{Additional useful lemmas}

\begin{lemma}
\label{lem:renyi_hellinger}
Let $I = -2 \log \left( \sqrt{P_0 Q_0} + \int \sqrt{(1-P_0)(1-Q_0) p(x) q(x)} dx \right)$ and let $H = (\sqrt{P_0} - \sqrt{Q_0})^2 + \int \left( \sqrt{ (1-P_0) p(x)} - \sqrt{ (1-Q_0) q(x)} \right)^2 dx$. If $I < 1- \epsilon$, then we have that

\[
I = H(1+\eta)
\]

where $|\eta| \leq \frac{H}{4\epsilon}$. Therefore, we have that $I \rightarrow 0$ iff $H \rightarrow 0$ and that if $I \rightarrow 0$, $ I = H(1+o(1))$.

\end{lemma}

\begin{proof}

\begin{align*}
I &= -2 \log \left( \sqrt{P_0 Q_0} + \int \sqrt{(1-P_0)(1-Q_0) p(x) q(x)} dx \right) \\
  &= -2 \log \left( 1 - \frac{1}{2} \left( 
                (\sqrt{P_0} - \sqrt{Q_0})^2 + 
               \int (\sqrt{(1-P_0)p(x)} - \sqrt{(1-Q_0)q(x)} )^2 dx \right) \right)\\
 &= -2 \log \left(1 - \frac{1}{2} H \right) \\
  &= 2 \frac{1}{2} H (1 + \eta)
\end{align*}

where $|\eta| \leq \frac{H}{2 \epsilon}$. 
\end{proof}

\begin{lemma}
Suppose $x \geq 0$ and $1 \geq \epsilon > 0$, then we have that, for all $0 \leq x < 1-\epsilon$,
\[
\log (1 - x) = - (1 + \eta) x 
\]
where $| \eta| \leq \frac{x}{2 \epsilon}$
\end{lemma}

\begin{proof}
This follows by taking the Taylor expansion of $\log (1 - x)$ around $x = 0$.
\end{proof}

\begin{lemma}
\label{lem:sqrt_linearize}
Define $f(z) =  \frac{1 - \frac{z}{2} - \sqrt{ 1 - z}}{z} $ for $z \leq 1$ and $z \neq 0$ and define $f(0) = 0$. Then we have that,
\[
\left| f(z)  \right| \leq |z|
\]
for all $z \leq 1$.
\end{lemma}

\begin{proof}

Define $f(z) = \frac{1 - \frac{z}{2} - \sqrt{ 1 - z}}{z}$, where we set $f(0) = 0$. Note that $f$ is continuous. 

The derivative of $f$ is 
\[
f'(z) = - \frac{1}{z^2} - \frac{z - 2}{2 z^2 \sqrt{1-z}}
\]

It is straight forward to check that $f'(z) \geq 0$ for all $z < 1$ and that we can define $f'(0) = \frac{1}{4}$ such that $f'(z)$ is continuous.

Therefore, $f(z)$ is monotonic and maximized at $z = 1$, yielding $f(1) = 1/2$ and minimized at $\lim_{z \rightarrow -\infty} f(z) = -\frac{1}{2}$. 

Now we perform case analysis. Suppose $z < -1/2$, then $|f(z)| \leq \frac{1}{2} < |z|$.

Suppose $-1/2 \leq z \leq 1/2$. By Taylor expansion, we have

\[
\sqrt{1 - z} = 1 - \frac{1}{2} z - \frac{1}{8} z^2 - \frac{1}{16} z^3  - ... \frac{(n+1)!!}{2^n n!} z^n - ... 
\]


Therefore,
\begin{align*}
\left| \sqrt{1-z} - (1 - \frac{z}{2}) \right| &\leq
     \frac{1}{8} (|z|^2 + |z|^3 + ... ) \\
  &\leq \frac{1}{8} |z|^2 ( 1 + |z| + |z|^2 + ...) \\
  &\leq \frac{1}{8} |z|^2 \frac{1}{1 - |z|} \\
  &\leq \frac{1}{4} |z|^2 
\end{align*}

Therefore, $|f(z)| \leq \frac{1}{4} |z|$. 

Finally, suppose $z > 1/2$. Then, 

$|f(z)| \leq \frac{1}{2} < z$. 




\end{proof}

%%% Local Variables:
%%% mode: latex
%%% TeX-master: "../paper"
%%% End:
