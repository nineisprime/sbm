

\subsection{Proof of Proposition~\ref{prop:labeled_sbm_rate}}

The proof of Proposition~\ref{prop:labeled_sbm_rate} is quite involved. On a high level, we first show that the initialization stages output a set of rough clusterings $\tilde{\sigma}_1, \dots, \tilde{\sigma}_n$ such that the error $l(\tilde{\sigma}_u, \sigma_0)$ for any $u$ is bounded by a sequence $\gamma$ where $\gamma \rho_L^2 \rightarrow 0$, that is, the errors of the rough clusterings are small enough. Note that $\gamma \rho_L^2 \rightarrow 0$ also implies that $\gamma \rightarrow 0$. This step uses Propositions~\ref{prop:initialization_correctness} and~\ref{prop:spectral_analysis}.

Then, we look at a single $\tilde{\sigma}_u$ and show that because $l(\tilde{\sigma}_u, \sigma_0)$ is small, the estimate $\hat{P}_l, \hat{Q}_l$ that we construct from it are close to the true $P_l, Q_l$. This is done through Proposition~\ref{prop:estimation_consistency}. Since $\hat{P}_l, \hat{Q}_l$ are good estimators, we prove, in Proposition~\ref{prop:single_node_error_bound}, that we can use $\log \frac{\hat{P}_l}{\hat{Q}_l}$ derived from $\tilde{\sigma}_u$ to correctly identify the true cluster membership of $u$ with high probability. Finally, we analyze the consensus stage to show that we can at the end construct a single coherent clustering from all the $\tilde{\sigma}_u$'s.

Formal details are provided below, with proofs of supporting propositions appearing in succeeding subsections.

\begin{proof} (Proof of Proposition~\ref{prop:labeled_sbm_rate})\\

First we note that discarding color $l$ where $P_l \vee  Q_l \leq \frac{c}{n}$ does not affect the Renyi divergence $I$. To formalize this, define
$P' = \sum_{l \,:\, P_l \wedge Q_l \geq c/n} P_l$ and $P'_l = P_l/P'$ and likewise for $Q'$ and $Q'_l$. Define also $I' = - 2 \log \sum_{l \,:\, P_l, Q_l \geq c/n} \sqrt{P'_l Q'_l}$.

We claim that $I' = (1 + o(1)) I_L$. We have

\begin{align*}
I_L &= -2 \log \sum_l \sqrt{P_l Q_l} \\
  &\leq -2 \log \sum_{l \,:, P_l \wedge Q_l \geq c/n} \sqrt{P_l Q_l} \\
  &\leq -2 \log \sum_{l \,:\, P_l \wedge Q_l \geq c/n} \sqrt{P'_l Q'_l} - 2 \log \sqrt{P' Q'} \\
  &\leq -2 \log \sum_{l \,:\, P_l \wedge Q_l \geq c/n} \sqrt{P'_l Q'_l} - 2\log\left(1 - \frac{cL}{n}\right),
\end{align*}
where the last inequality follows because $P' \geq 1 - \sum_{l \,:\, P_l \leq c/n} P_l \geq 1 - \frac{cL}{n}$, and likewise for $Q'$. Since $\frac{n I_L}{L} \rightarrow \infty$, we have that $\frac{cL}{n} = o(I_L) = o(1)$. The claim follows. 

Now we prove the main result. Since $\frac{ nI}{L \rho_L^2} \rightarrow \infty$, Proposition~\ref{prop:initialization_correctness} shows that the first initialization stage of our algorithm selects a color $l^*$ that satisfies $\frac{n (P_{l^*} - Q_{l^*})^2}{ (P_{l^*} \vee Q_{l^*}) \rho^2_L } \rightarrow \infty$ with probability at least $1 - Ln^{3 + \delta_p}$. Let us denote this event by $E_1$.

We then apply our analysis of spectral clustering (Proposition~\ref{prop:spectral_analysis}) on each of the clusterings $\tilde{\sigma}_1, \dots, \tilde{\sigma}_n$ and a union bound to show that $\max_u l( \tilde{\sigma}_u, \sigma_0) \leq \gamma$ with probability at least $1 - n^3$, where 
\[
\gamma = C K^2 \frac{ P_{l^*} \vee Q_{l^*}}{ n ( P_{l^*} - Q_{l^*} )^2} 
\]
Let us denote this event by $E_2$. 

Conditioned on $E_1$ and $E_2$ described above, we have that $\gamma \rho_L^2 \rightarrow 0$. Thus, we can apply Proposition~\ref{prop:estimation_consistency} on each of the rough clustering $\tilde{\sigma}_u$'s and show that the conclusion of Proposition~\ref{prop:estimation_consistency} holds simultaneously for all $\tilde{\sigma}_u$ with probability at least $1 - Ln^{2 + \delta_p}$. Furthermore, the $\eta$ that appears in Proposition~\ref{prop:estimation_consistency} ($\eta = 4 \left( \sqrt{2 (5 + \delta_p) K \gamma \log \frac{K}{\gamma} } + \frac{5 + \delta_p}{c} K \gamma \log \frac{K}{\gamma} + \gamma K \right)$) satisfies $|\eta \rho_L| \rightarrow 0$. Let us denote this event by $E_3$.

Now we condition on $E_3$ and look at the refinement and consensus stage of the algorithm. Let $\hat{\sigma}_u$ be the intermediate clustering on $n$ nodes created in the refinement stage. By construction of $\hat{\sigma}_u$, the error rate of $\hat{\sigma}_u$ is at most $ \gamma + \frac{1}{n}$ and thus, $l(\hat{\sigma}_u, \sigma_0) < \frac{1}{8 \beta K}$ for small enough $\gamma$. Let $\pi_u \in S_K$ denote the permutation such that $d(\pi_u(\hat{\sigma}_u), \sigma_0) < \frac{n}{8\beta K}$. 

It is also clear that for any $u$, the minimum cluster size of the clustering $\hat{\sigma}_u$ is at least $\frac{n}{\beta K} - (n \gamma + 1) \geq \frac{ n(1 - \gamma k - 1/n)}{\beta K} \geq \frac{n}{2\beta K}$ for small enough $\gamma$. Therefore, we know by Lemma~\ref{lem:consensus_uniqueness} that $\pi_u$ is the permutation that minimizes the $l(\hat{\sigma}_u, \sigma_0)$. 

Conditioned on $E_3$, we can apply proposition~\ref{prop:single_node_error_bound} to conclude that
\[
 P( \hat{\sigma}_u(u) \neq \pi^{-1}_u(\sigma_0(u)) \given E_3) \leq (K-1) \exp\left( - (1+o(1)) \frac{n I_L}{\beta K} \right)
\]


By a triangle inequality argument, we have that $l(\tilde{\sigma}_u, \tilde{\sigma}_1) \leq 2\gamma + 2/n < \frac{1}{4\beta K}$. Thus, we can apply lemma~\ref{lem:consensus_analysis} on the pair $(\tilde{\sigma}_1, \tilde{\sigma}_u)$ to show that the consensus function $\xi_u$ is the permutation that minimizes 
$d(\xi_u(\tilde{\sigma}_1), \tilde{\sigma}_u)$.\\

We claim then that $\pi_1 = \pi_u \cdot \xi_u$. We know that $d(\pi_1(\tilde{\sigma}_1), \sigma_0) < \frac{n}{8 \beta K}$ and that $d(\pi_u(\tilde{\sigma}_u), \sigma_0) < \frac{n}{8 \beta K}$. Therefore, $d(\tilde{\sigma}_1, \pi_u( \pi_1^{-1}(\tilde{\sigma}_u))) < \frac{n}{4 \beta K}$. Since the minimum cluster size of both $\tilde{\sigma}_1$ and $\tilde{\sigma}_u$ is $\frac{n}{2 \beta K}$, we have that $\pi_u^{-1} \cdot \pi_1 = \xi_u$ by Lemma~\ref{lem:consensus_uniqueness}.

Let $\hat{\sigma}$ be the output of the consensus stage. We have that
\begin{align*}
P( \hat{\sigma}(u) \neq \pi_1^{-1} ( \sigma_0(u)) \given E_3) &= P( \hat{\sigma}(u) \neq \xi_u^{-1} \circ \pi_u^{-1} (\sigma_0(u)) \given E_3)\\
   &= P( \hat{\sigma}_u(u) \neq \pi_u^{-1} (\sigma_0(u)) \given E_3) 
\end{align*}

Now we are almost done with the proof.  
\begin{align*}
P( \hat{\sigma}(u) \neq \pi_1^{-1} ( \sigma_0(u)) ) &\leq (K-1) \exp\left( -(1 - \eta') \frac{n I_L}{\beta K} \right) + P(E_3^c \given E_2, E_1) + P(E_2^c) + P(E_3^c) \\
    &\leq (K-1) \exp\left( -(1 - \eta') \frac{n I_L}{\beta K} \right) + Ln^{-(2 + \delta_p)} + n^{-3} + Ln^{-3} \\
    &\leq (K-1) \exp\left( -(1 - \eta') \frac{n I_L}{\beta K} \right) + n^{-(1 + \delta_p)} + n^{-3} + n^{-2} \\
    &\leq  (K-1) \exp\left( -(1 - \eta') \frac{n I_L}{\beta K} \right) + n^{-(1 + \delta_p)}
\end{align*}

where $\eta'$ is some $o(1)$ sequence and where the third inequality follows because we take $L < n$. 

We take then $\eta'' = \eta' + \beta \sqrt{ \frac{K}{ n I_L} } = o(1)$. 

First, suppose that $ (K-1) \exp\left( -(1 - \eta') \frac{n I_L}{\beta K} \right) \geq n^{-(1+\delta_p/2)} $. 

\begin{align*}
& P\left\{ l(\hat{\sigma}, \sigma_0) >  (K-1) \exp\left( -(1 - \eta'') \frac{n I_L}{\beta K} \right) \right\} \\
 &\leq \frac{1}{ (K-1) \exp\left( -(1 - \eta'') \frac{n I_L}{\beta K} \right) } \frac{1}{n} \sum_{u=1}^n P(\hat{\sigma}(u) \neq \pi_1^{-1}(\sigma_0(u))) \\
 &\leq \exp\left\{ -(\eta'' - \eta') \frac{n I_L}{\beta K} \right\} + \frac{ C n^{-(1 + \delta_p)} }{  (K-1) \exp\left( -(1 - \eta') \frac{n I_L}{\beta K} \right)  }\\
 &\leq \exp \left\{ - \sqrt{ \frac{n I_L}{K} } \right\} + n^{-\delta_p/2} = o(1)
\end{align*}

Now, suppose that $ (K-1) \exp\left( -(1 - \eta') \frac{n I_L}{\beta K} \right) \leq n^{-(1 + \delta_p/2)}$.

\begin{align*}
& P \left\{ l(\hat{\sigma}, \sigma_0) >  (K-1) \exp\left( -(1 - \eta') \frac{n I_L}{\beta K} \right)  \right\} \\
& = P( l(\hat{\sigma}, \sigma_0) > 0 ) \\
& \leq \sum_{u=1}^n P( \hat{\sigma}(u) \neq \pi_1^{-1}(\sigma_0(u))) \\
&\leq n  (K-1) \exp\left( -(1 - \eta') \frac{n I_L}{\beta K} \right)  + n^{-\delta_p} \leq n^{-\delta_p /2} = o(1)
\end{align*}

\end{proof}

\subsubsection{Analysis of estimation error of $\hat{P}_l$ and $\hat{Q}_l$}


\begin{proposition}
\label{prop:estimation_consistency}
Let $A_L$ be a labeled network with true clustering $\sigma_0$. 

Suppose $\sigma$ is a random initial clustering and let us condition on the supposition that $\sigma$ has error rate at most $\gamma$, that is, $l(\sigma, \sigma_0) \leq \gamma $. \\

Let $\Delta_l = | P_l - Q_l |$. Let $\hat{P}_l = \frac{\sum_{u \neq v \,:\, \sigma(u)=\sigma(v)} \mathbf{1}(A_{uv} = l) }
                      {\sum_{u \neq v \,:\, \sigma(u) = \sigma(v)} 1}$ and
    $\hat{Q}_l = \frac{\sum_{u \neq v \,:\, \sigma(u) \neq \sigma(v)} \mathbf{1}(A_{uv} = l) }
                      {\sum_{u \neq v \,:\, \sigma(u) \neq \sigma(v)} 1}$ be the MLE of $P_l$ and $Q_l$ based on $\sigma$. Let $C_{thresh}$ be an absolute constant and let $\delta_p$ be a positive, fixed, and arbitrarily small real number. Let $c$ be an absolute positive constant. \\

Then, with probability at least $1 - L n^{-(3 + \delta_p)}$, the following event happens:


For all $l$ such that $P_l \vee Q_l \geq \frac{c}{n}$ , 
\begin{itemize}
\item[Case 1] if $\frac{n \Delta_l^2}{P_l \vee Q_l} \geq C_{thresh}$, then
\begin{align*}
 | \hat{P}_l - P_l | &\leq \eta \Delta_l \\
 | \hat{Q}_l - Q_l | &\leq \eta \Delta_l 
\end{align*}
\item[Case 2] if $\frac{n \Delta_l^2}{P_l \vee Q_l} \leq C_{thresh}$, then
\begin{align*}
 | \hat{P}_l - P_l | &\leq \eta \sqrt{ \frac{P_l \vee Q_l}{n}} \\
 | \hat{Q}_l - Q_l | &\leq \eta \sqrt{ \frac{P_l \vee Q_l}{n}}
\end{align*}
\end{itemize}
Where $\eta = 4 \left( \sqrt{2 (5 + \delta_p) K \gamma \log \frac{K}{\gamma} } + \frac{5 + \delta_p}{c} K \gamma \log \frac{K}{\gamma} + \gamma K \right)$ is independent of the color $l$. 
\end{proposition}

\begin{proof}

We first analyze the estimation error for a fixed $\sigma$ and then take a union bound over all $\sigma$ that satisfies $d_H(\sigma, \sigma_0) \leq \gamma n$. 

Note that there are at most $\binom{n}{\gamma n} K^{\gamma n}$ possible assignments $\sigma$'s that satisfy the error rate constraint. 

\begin{align*}
\log \binom{n}{\gamma n} K^{\gamma n} & =
  \log \left( \frac{ n(n-1) ...(n-\gamma n+1) }{(\gamma n)!} \right) + \gamma n \log K \\
 & \leq \log \left( \frac{ n^{\gamma n} e^{\gamma n} }
     { (\gamma n)^{\gamma n} } \frac{1}{\sqrt{2\pi \gamma n}} \right) + \gamma n \log K \\
 & \leq \log \left( \frac{ e^{\gamma n} }{\gamma^{\gamma n}} \right) - \frac{1}{2} \log 2 \pi \gamma n + \gamma n \log K \\
 & \leq \gamma n \log \frac{e}{\gamma}  + \gamma n \log K \\
 & \leq 2 \gamma n \log \frac{K}{\gamma}
\end{align*}


Next, we bound the bias of $\hat{P}_l$.

Our estimator of $P_l$ is 
\[
\hat{P_l} = \frac{ \sum_{u \neq v \,:\, \sigma(u) = \sigma(v)} \mathbf{1}(A_{uv} = l) }{
                   \sum_{u \neq v \,:\, \sigma(u) = \sigma(v)} }
\]

$\E \hat{P}_l$ is a convex combination of $P_l, Q_l$. 
\begin{align}
\E \hat{P_l} &= 
   \frac{ \sum_{u \neq v \,:\, \sigma(u) = \sigma(v)} 
             \mathbf{1}(\sigma_0(u) = \sigma_0(v) ) P_l + 
               \mathbf{1}(\sigma_0(u) \neq \sigma_0(v)) Q_l }{
                   \sum_{u \neq v \,:\, \sigma(u) = \sigma(v)} 1 } \nonumber \\
  &= (1 - \lambda) P_l + \lambda Q_l  = P_l + \lambda (Q_l - P_l) \label{eqn:bias_simple_bound}
\end{align}
for $\lambda = \frac{\sum_{u \neq v \,:\, \sigma(u) = \sigma(v)} 
     \mathbf{1}(\sigma_0(u) \neq \sigma_0(v)) }{\sum_{u \neq v \,:\, \sigma(u) = \sigma(v)} 1}$.

Thus, we have that 
\[
|\E \hat{P}_l - P_l | \leq \lambda |Q_l - P_l|
\]

We need to upper bound $\lambda$. observe that
\begin{align*}
\lambda 
  &= \frac{\sum_{u \neq v \,:\, \sigma(u) = \sigma(v) }\mathbf{1}(\sigma_0(u) \neq \sigma_0(v)) }{\sum_{u \neq v} \mathbf{1}(\sigma(u) = \sigma(v)) } 
      \\
  &= 
   \frac{\sum_k \sum_{u \neq v \,:\, \sigma(u)=\sigma(v)=k} \mathbf{1}(\sigma_0(u) \neq \sigma_0(v))}{\sum_k \hat{n}_k (\hat{n}_k-1)} 
      \\
  &\leq \frac{\sum_k \sum_{u \neq v \,:\, \sigma(u)=\sigma(v)=k} 
       \mathbf{1}( \neg (\sigma_0(u) = \sigma_0(v) =k) )}{\sum_k \hat{n}_k (\hat{n}_k-1)} 
      \\ 
  &\leq \frac{ \sum_k \sum_{u \neq v \,:\, \sigma(u)=\sigma(v)=k} \mathbf{1}(\sigma_0(v)) \neq k) + \mathbf{1}(\sigma_0(u) \neq k)}
            {\sum_k \hat{n}_k (\hat{n}_k - 1)} 
\end{align*}
Define $\gamma_k = \frac{1}{n} \sum_{u \,:\, \sigma(u)=k} \mathbf{1}(\sigma_0(u) \neq k)$ as the error rate within the estimated cluster $k$, and define $\hat{n}_k = \sum_u \mathbf{1}(\sigma(u) = k)$. Then, we have that $\sum_k \gamma_k = \gamma$ and also $\sum_{u \,:\, \sigma(u) = k} \sum_{v \,:\, \sigma(v) = k} \mathbf{1}(\sigma_0(v) \neq k) = \gamma_k n \hat{n}_k$. We continue the bound: 
\begin{align*}
\lambda  
  &\leq \frac{ \sum_k 2 \gamma_k n \hat{n}_k }{\sum_k \hat{n}_k(\hat{n}_k - 1)} 
     \\
  &= \frac{n}{\sum_k \hat{n}_k (\hat{n}_k - 1) } \sum_k 2 \gamma_k \hat{n}_k 
     \\
  &\leq \frac{K}{n-K} n \sum_k 2 \gamma_k \frac{\hat{n}_k}{n} \\
  &\leq 4 \gamma K
\end{align*}

In the first inequality, we used the fact that $\sum_k \frac{\hat{n}_k }{n} (\hat{n}_k - 1) = n \sum_k \left( \frac{\hat{n}_k}{n} \right)^2 - 1 \geq \frac{n}{K} - 1$ since $\sum_k \frac{\hat{n}_k}{n} = 1$. In the last inequality, we used the assumption that $K < \frac{n}{2}$. 

We then have an upper bound for $\lambda \leq 4 \gamma K$. Therefore, we have that

\[
|\E \hat{P}_l - P_l | \leq 4 \gamma K \Delta_l
\]

where $\Delta_l  = |Q_l - P_l|$. To simplify presentation, we define $\eta_1 = 4 \gamma K$ so that
\[
|\E \hat{P}_l - P_l | \leq \eta_1 \Delta_l
\]
From this it is clear that $\eta_1$ becomes arbitrarily small if $\gamma$ is made arbitrarily small. 

Having bounded the bias, we can now bound the variance.


Let $ \tilde{A}_{uv} = \mathbf{1}(A_{ij} = l)$. Then, by Bernstein's inequality,
\[
P\left( \left| \sum_{u,v\,:\, \sigma(u) = \sigma(v)} (\tilde{A}_{uv} - \E \tilde{A}_{uv} ) \right|  > t 
 \right) \leq 2 \exp\left( 
    - \frac{t^2}{ 2 \sum_{u,v \, \sigma(u) = \sigma(v)} \E \tilde{A}_{uv}  + \frac{2}{3}t } 
\right)
\]

We first bound $\sum_{u,v \, \sigma(u) = \sigma(v)} \E \tilde{A}_{uv}$:
\begin{align*}
\sum_{u,v \, \sigma(u) = \sigma(v)} \E \tilde{A}_{uv} &=
  \sum_k \hat{n}_k (\hat{n}_k - 1) \E \hat{P}_l \\
 &\leq (P_l \vee Q_l) \sum_k \hat{n}_k (\hat{n}_k - 1) \quad 
  \trm{(by Equation~\ref{eqn:bias_simple_bound})}
\end{align*}

Therefore,
\[
P\left( \left| \sum_{u,v\,:\, \sigma(u) = \sigma(v)} (\tilde{A}_{uv} - \E \tilde{A}_{uv} ) \right|  > t 
 \right) \leq 2 \exp\left( 
    - \frac{t^2}{ 2 (P_l \vee Q_l) \sum_k \hat{n}_k (\hat{n}_k - 1)  + \frac{2}{3}t } 
\right)
\]

Our goal is to choose an appropriate $t$ such that the probability is upper bounded by $\exp( - 2 \gamma n \log \frac{K}{\gamma} - (3+\delta_p) \log n )$. 

We choose the following $t$
\begin{align*}
t^2 &= 4 \left\{  \left( 2 (P_l \vee Q_l) \sum_k \hat{n}_k (\hat{n}_k - 1) \right) 
      \left( 
    2 \gamma n \log \frac{K}{\gamma} + (3+\delta_p) \log n \right) \right\} 
      \vee 
   4  \left\{
   \left( 2 \gamma n \log \frac{K}{\gamma} + (3 + \delta_p) \log n \right)^2 \right\} 
\end{align*}

We now verify that regardless of which term among $\{ 2 (P_l \vee Q_l) \sum_k \hat{n}_k (\hat{n}_k - 1) ,\,  2 \gamma n \log \frac{K}{\gamma} + (3+\delta_p) \log n\}$ is larger, the probability term is at most $2 \exp\left( -  \left( 2 \gamma n \log \frac{K}{\gamma} + (3+\delta_p) \log n \right) \right)$. Let $A = 2(P_l \vee Q_l)\sum_k \hat{n}_k (\hat{n}_k - 1)$ and
 $B =  2 \gamma n \log \frac{K}{\gamma} + (3+\delta_p) \log n$. 

\begin{itemize}
\item Suppose $A \geq B$, then $t^2 = 4AB$ and the probability term is at most
    $2 \exp \left( - \frac{4 AB}{A + \frac{4}{3} \sqrt{AB}} \right)  \leq 2 \exp \left( - \frac{4 AB}{A + \frac{4}{3} A} \right) \leq  2 \exp( - B )$ 
\item Suppose $A \leq B$, then $t^2 = 4B^2$ and the probability term is at most
       $2 \exp \left( - \frac{4 B^2}{A + \frac{4}{3} B} \right) \leq 2 \exp \left( - \frac{4 B^2}{ B + \frac{4}{3} B} \right) \leq 2 \exp( - B)$. 
\end{itemize}

Thus, with probability at most $\exp\left( -  \left( 2 \gamma n \log \frac{K}{\gamma} + (3+\delta_p) \log n \right) \right)$,

\begin{align*}
| \hat{P}_l - \E \hat{P}_l | =
\frac{\sum_{u,v \, \sigma(u) = \sigma(v)} (\tilde{A}_{uv} - \E \tilde{A}_{uv} ) }{
  \sum_{u,v} \mathbf{1}( \sigma(u) = \sigma(v) ) } &> 
  \frac{t}{\sum_{u,v} \mathbf{1}(\sigma(u) = \sigma(v)) } \\
\end{align*}

Now we derive a more manageable upper bound for $t$:

Note that 
\[
t^2 \leq 4 \left\{ \sqrt{  \left(2 (P_l \vee Q_l)  \sum_k \hat{n}_k (\hat{n}_k - 1) \right) 
               \left( 2 \gamma n \log \frac{K}{\gamma} + (3+\delta_p) \log n \right)} 
           + 
          \left(  2 \gamma n \log \frac{K}{\gamma} + (3+\delta_p) \log n \right) \right\}^2
\]

Note that we can without loss of generality assume $\gamma \geq \frac{1}{n}$. We then have that $\gamma n \log \frac{1}{\gamma} \geq \log n$. Thus, $2 \gamma n \log \frac{K}{\gamma} + (3 + \delta_p) \log n \leq (5 + \delta_p) \gamma n \log \frac{K}{\gamma}$.


\begin{align*}
 \frac{t}{\sum_{u,v} \mathbf{1}(\sigma(u) = \sigma(v)) } &\leq
  2 \frac{  \sqrt{  \left( 2 (P_l \vee Q_l) \sum_k \hat{n}_k (\hat{n}_k - 1) \right) 
                 \left( 2 \gamma n \log \frac{K}{\gamma} + (3+\delta_p) \log n \right) }  
          + 
               \left(  2 \gamma n \log \frac{K}{\gamma} + (3+\delta_p) \log n\right)  }
        { \sum_{u,v} \mathbf{1}(\sigma(u) = \sigma(v) ) }  \\
  &\leq 2 \frac{\sqrt{2 (P_l \vee Q_l) (5+\delta_p) \gamma n \log \frac{K}{\gamma}}}
             {\sqrt{ \sum_k \hat{n}_k (\hat{n}_k - 1)}} + 
        2  \frac{(5+\delta_p) \gamma n \log \frac{K}{\gamma}}
             {\sum_k \hat{n}_k (\hat{n}_k - 1)} \\
 &\leq  2 \frac{ \sqrt{2 (P_l\vee Q_l)} \sqrt{ (5+\delta_p) \gamma K \log \frac{K}{\gamma}} }
           {n - K} + 
       2 \frac{(5+\delta_p) \gamma K \log \frac{1}{\delta}}{n - K} \\
 &\leq 4 \sqrt{ \frac{P_l \vee Q_l}{n} } \sqrt{(5+\delta_p) K \gamma \log \frac{K}{\gamma}} + 
       4 \frac{(5+\delta_p) K \gamma \log \frac{K}{\gamma}}{n} 
\end{align*}

For the second to last inequality, we used the fact that $\sum_k (\hat{n}_k - 1) \frac{\hat{n}_k}{n} = n \sum_k \left( \frac{\hat{n}_k}{n} \right)^2 - 1 \geq \frac{n}{K} - 1 $ because $\frac{\hat{n}_k}{n}$ sums to 1. In the last inequality, we used the assumption that $n-K \geq \frac{n}{2}$. 

To further simplify the expression, we have that $P_l \vee Q_l \geq \frac{c}{n}$ and thus, $\sqrt{ \frac{P_l \vee Q_l}{n} } \geq \frac{c}{n}$. Therefore, we have that, with probability at least $1 - \exp( -C_1 \gamma n \log \frac{K}{\gamma} - (3 + \delta) \log n)$, 
\begin{align}
| \hat{P}_l - \E \hat{P}_l | \leq \sqrt{ \frac{P_l \vee Q_l}{n} } \eta_2 \label{eqn:variance_bound}
\end{align}
where $\eta_2 = 4 \left( \sqrt{2 (5+\delta_p) K \gamma \log \frac{K}{\gamma}} 
      + \frac{(5+\delta_p)}{c} K \gamma \log \frac{K}{\gamma} \right)$. It is clear that $\eta_2$ can be made arbitrarily small by taking $\gamma K \log K$ to be arbitrarily small. 

 Taking the union bound across all clusterings with error $\gamma$ and across all colors, we have that the probability of (\ref{eqn:variance_bound}) holding simultaneously for all colors $l$ is at least $1 - L n^{-(3+\delta_p)}$. 

\end{proof}





\subsubsection{Analysis of probability of error for a single node}


As a first step toward proving proposition~\ref{prop:labeled_sbm_rate}, we first analyze the probability of error for a single node $u$. 

\begin{proposition}
\label{prop:single_node_error_bound}

Let node $u$ be arbitrarily fixed and suppose that $\frac{n I_L}{L} \rightarrow \infty$ and that for all $l$, $\frac{1}{\rho_L} \leq \frac{P_l}{Q_l} \leq \rho_L$.  Suppose also that $\frac{1}{2} \sum_l (\sqrt{P_l} - \sqrt{Q_l})^2 \leq \frac{1}{2}$. Conditioning on the event that the result of Proposition~\ref{prop:estimation_consistency} holds with a sequence $\eta$ that satisfies $\eta \rho_L \rightarrow 0$, we have that, with probability at least $1 - (K-1)\exp \left( - (1 - o(1)) \frac{n}{\beta K} I_L \right)$, the following event holds:
\[
\sigma_0(u) = \argmax_k \sum_{v\,: \sigma_u(v)=k} \sum_l \log \frac{\hat{P}_l}{\hat{Q}_l} \mathbf{1}(A_{uv} = l) 
\]

\end{proposition}

\begin{proof}

Throughout the proof, we let $\eta'$ denote a sequence that converges to 0 and let $C$ denote a $\Theta(1)$ sequence. Their value could change from line to line. \\

Let $C_{thresh}$ be an absolute constant as defined in Proposition~\ref{prop:estimation_consistency}.

First, define $L_1 = \{ l \,:\, n \frac{\Delta_l^2}{P_l \vee Q_l} \geq C_{thresh} \}$. Then we claim that $C \sum_{l \in L_1} \frac{\Delta_l^2}{P_l \vee Q_l} = I_L ( 1 - \eta' )$. To see this, observe first that

\begin{align*}
I_L &= -2 \log \sum_l \sqrt{P_l Q_l} \\
  &= C \sum_l (\sqrt{P_l} - \sqrt{Q_l} )^2 \\
  &= C \sum_l \frac{\Delta_l^2}{(\sqrt{P_l} + \sqrt{Q_l})^2} \\
  &= C \sum_l \frac{\Delta_l^2}{P_l \vee Q_l} 
\end{align*}

Therefore, we have that
\begin{align*}
C \sum_{l \in L_1} \frac{\Delta_l^2}{P_l \vee Q_l} &= I_L - 
 C \sum_{l \notin L_1} \frac{\Delta_l^2}{P_l \vee Q_l} \\
 &\geq I_L - C_{\epsilon, c_1, c_2} \sum_{l \notin L_1} \frac{C_{thresh}}{n} \\
 &\geq I_L - C_{\epsilon, c_1, c_2} \frac{ L C_{thresh}}{n} \\
 &\geq I_L - \eta' I_L
\end{align*}

The first inequality follows from the definition of $L_1$. The third inequality follows because $\frac{n I_L}{L} \rightarrow \infty$ by assumption. \\


Now we proceed onto the main proof. Suppose without the loss of generality that $\sigma_0(u) = 1$.  We want to then control the probability that for some cluster $k$, 
\begin{align}
\sum_{v\,:\, \sigma_u(v)=k} \sum_l \log \frac{\hat{P}_l}{\hat{Q}_l} \mathbf{1}(A_{uv} = l) 
&\geq 
 \sum_{v\,:\, \sigma_u(v)=1} \sum_l \log \frac{\hat{P}_l}{\hat{Q}_l} \mathbf{1}(A_{uv} = l) 
  \quad \trm{(iff)}  \nonumber \\
\sum_{v \,:\, \sigma_u(v) = k} \bar{A}_{uv} - \sum_{v\,:\, \sigma_u(v) = 1} \bar{A}_{uv} 
\label{eqn:bad_event1}
&\geq 0 
\end{align}
where $\bar{A}_{uv} \equiv \sum_l \log \frac{\hat{P}_l}{\hat{Q}_l} \mathbf{1}(A_{uv} = l)$. 

Define $m_1 = |\{ v \,:\, \sigma_u(v) = 1 \}|$ and $m_k = | \{ v \,:\, \sigma_u(v) = k \}|$ as the size of clusters $m_1, m_k$ under $\sigma_u$. Define $m_k' = \{ v \,:\, \sigma_u(v) = k,\, \sigma_0(v) = k \}$, $m_1' = \{ v \,:\, \sigma_u(v) = 1 ,\, \sigma_0(v) = 1\}$ as the points correctly clustered by $\sigma_u$. 

With these definitions, the probability of the bad event Equation~\ref{eqn:bad_event1} is upper bounded by the probability of the following:

\begin{align*}
\left( \sum_{i=1}^{m_k'} \tilde{Y}_i + \sum_{i=1}^{m_k - m'_k} \tilde{X}_i \right) - 
\left( \sum_{i=1}^{m_1'} \tilde{X}_i + \sum_{i=1}^{m_1 - m_1'} \tilde{Y}_i  \right) &\geq 0  \quad \trm{(iff)} \\
\exp( t \left( \sum_{i=1}^{m_k'} \tilde{Y}_i + \sum_{i=1}^{m_k - m_k'} \tilde{X}_i - 
     \sum_{i=1}^{m_1'}  \tilde{X}_i - \sum_{i=1}^{m_1 - m_1'} \tilde{Y}_i  \right) ) &\geq 1 
\end{align*}

 where $\tilde{X}_i = \log \frac{\hat{P}_l}{\hat{Q}_l}$ with probability $P_l$ and $\tilde{Y}_i = \log \frac{\hat{P}_l}{\hat{Q}_l}$ with probability $Q_l$. 



\begin{align*}
& P \left( \exp( t \left( \sum_{i=1}^{m_k'} \tilde{Y}_i + \sum_{i=1}^{m_k - m_k'} \tilde{X}_i- 
     \sum_{i=1}^{m_1'}  \tilde{X}_i - \sum_{i=1}^{m_1 - m_1'} \tilde{Y}_i  \right) ) \geq 1 \right) \\ 
&\leq \E \left( 
\exp( t \left( \sum_{i=1}^{m_k'} \tilde{Y}_i + \sum_{i=1}^{m_k - m_k'} \tilde{X}_i - 
     \sum_{i=1}^{m_1'}  \tilde{X}_i - \sum_{i=1}^{m_1 - m_1'} \tilde{Y}_i  \right) )
 \right) \\ 
&\leq \E \left( \E[\exp( t \tilde{Y}_i) \given \hat{P}_l, \hat{Q}_l] \right)^{m_k'} 
      \left( \E[ \exp(t \tilde{X}_i )\given \hat{P}_l, \hat{Q}_l] \right)^{m_k - m_k'}  
    \left( \E[ \exp( - t \tilde{X}_i)\given \hat{P}_l, \hat{Q}_l] \right)^{m_1'} 
    \left( \E[ \exp( -t \tilde{Y}_i )\given \hat{P}_l, \hat{Q}_l] \right)^{m_1 - m_1'} \\
&\leq\E \left( \sum_l e^{t \log \frac{\hat{P}_l}{\hat{Q}_l} } Q_l \right)^{m_k'}  
      \left( \sum_l e^{t \log \frac{\hat{P}_l}{\hat{Q}_l}} P_l \right)^{m_k - m_k'} 
      \left( \sum_l e^{- t \log \frac{\hat{P}_l}{\hat{Q_l}} } P_l \right)^{m_1'}
     \left( \sum_l e^{-t \log \frac{\hat{P}_l}{\hat{Q}_l}} Q_l \right)^{m_1 - m_1'}
\end{align*}

We will set $t = \frac{1}{2}$, in which case, we have:
\begin{align}
& \left( \sum_l \sqrt{\frac{\hat{P}_l}{\hat{Q}_l} } Q_l \right)^{m_k'}
 \left( \sum_l \sqrt{\frac{\hat{P}_l}{\hat{Q}_l} } P_l \right)^{m_k - m_k'}
 \left( \sum_l \sqrt{\frac{\hat{Q}_l}{\hat{P}_l} } Q_l \right)^{m_1 - m_1'}
       \left( \sum_l \sqrt{\frac{\hat{Q_l}}{\hat{P}_l} } P_l \right)^{m_1'} \nonumber \\
=&  \left( \frac{\sum_l \sqrt{\frac{\hat{P}_l}{\hat{Q}_l} } P_l}
                {\sum_l \sqrt{\frac{\hat{P}_l}{\hat{Q}_l} } Q_l}  \right)^{m_k - m_k'}
 \left( \frac{ \sum_l \sqrt{\frac{\hat{Q}_l}{\hat{P}_l} } Q_l}
             { \sum_l \sqrt{\frac{\hat{Q}_l}{\hat{P}_l} } P_l} \right)^{m_1 - m_1'}  
   \label{eqn:excess_error_term} \\
 & \left( \sum_l \sqrt{ \frac{\hat{P}_l}{\hat{Q}_l}} Q_l \right)^{m_k} 
    \left( \sum_l \sqrt{\frac{\hat{Q_l}}{\hat{P}_l} } P_l \right)^{m_1}
   \label{eqn:Ihat_error_term} 
\end{align} 

We will bound term~\ref{eqn:excess_error_term} and \ref{eqn:Ihat_error_term} separately. Loosely speaking, we will show that term~\ref{eqn:excess_error_term} is bounded in magnitude by $\exp( o(I_L) \frac{n}{K} )$ and that term~\ref{eqn:Ihat_error_term} is bounded by $\exp( - \frac{n}{K} (1 + o(1) I_L)$. 


\textbf{Bound for Term~\ref{eqn:excess_error_term}.} 

Now, we can bound term~\ref{eqn:excess_error_term}:
\begin{align*}
\left| 1 -  \frac{\sum_l \sqrt{\frac{\hat{P}_l}{\hat{Q}_l} } P_l}
                {\sum_l \sqrt{\frac{\hat{P}_l}{\hat{Q}_l} } Q_l}  \right|
 &= \left| \frac{ \sum_l \sqrt{ \frac{\hat{P}_l}{\hat{Q}_l}} (P_l - Q_l) }
     { \sum_l \sqrt{ \frac{\hat{P}_l}{\hat{Q}_l}} Q_l } \right| \\
&\leq \frac{8}{\sum_l \sqrt{P_l Q_l}} 
     \left| \sum_l \sqrt{ \frac{\hat{P}_l}{\hat{Q}_l} }(P_l - Q_l) \right| \\
&\leq 16 \left|  \sum_{l} \left( \sqrt{ \frac{\hat{P}_l}{\hat{Q}_l} } - 1 \right) (P_l - Q_l)  \right| \\
&\leq 16 \left| 
     \sum_{l \in L_1} \left( \sqrt{\frac{\hat{P}_l}{\hat{Q}_l}} - 1 \right)(P_l - Q_l) 
     \right| + 16 \sum_{l \notin L_1} \left| \sqrt{ \frac{\hat{P}_l}{\hat{Q}_l}} - 1 \right| |P_l - Q_l| \\
&\leq 16 \sum_{l \in L_1} \frac{\Delta^2_l}{Q_l}(1+ \eta') + 
      \sum_{i \notin L_1} 2 C_{thresh} \frac{\Delta_l}{\sqrt{n (P_l \vee Q_l)}} \\
&\leq C I_L (1 + \eta') + 2 C_{thresh}^2 \frac{L}{n} \\
&\leq C I_L (1 + \eta')
\end{align*}


Where the second inequality follows from lemma~\ref{lem:sqrt_ratio_times_ql},
second to last inequality follows under the assumption that $\sum_l \sqrt{P_l Q_l} \geq \frac{1}{2}$, and the last inequality follows from Lemma~\ref{lem:sqrt_ratio_pl_ql_minus_1}. 

Identical analysis shows that
\[
\left| 1 - \frac{ \sum_l \sqrt{\frac{\hat{Q}_l}{\hat{P}_l} } Q_l}
             { \sum_l \sqrt{\frac{\hat{Q}_l}{\hat{P}_l} } P_l} \right| 
= O(I_L) 
\]

Now, we note that $\exp( | 1 - x | ) \geq |x|$. 
Therefore, term~\ref{eqn:excess_error_term} can be bounded as
\begin{align*}
&  \left( \frac{\sum_l \sqrt{\frac{\hat{P}_l}{\hat{Q}_l} } P_l}
                {\sum_l \sqrt{\frac{\hat{P}_l}{\hat{Q}_l} } Q_l}  \right)^{m_k - m_k'}
 \left( \frac{ \sum_l \sqrt{\frac{\hat{Q}_l}{\hat{P}_l} } Q_l}
             { \sum_l \sqrt{\frac{\hat{Q}_l}{\hat{P}_l} } P_l} \right)^{m_1 - m_1'}  \\
&\leq \exp( O(I_L) (m_k - m_k' + m_1 - m_1') ) \\
&\leq \exp( O(I_L) \gamma n) \\
&\leq \exp\left( \frac{n}{K} o(I_L) \right) \quad \trm{(since $\gamma K \rightarrow 0$)}
\end{align*}

\textbf{Bound for Term~\ref{eqn:Ihat_error_term}.}


Define 
$\hat{I} = - \log \left( \sum_l \frac{\hat{P}_l}{\hat{Q}_l} Q_l \right) \left( \sum_l \frac{\hat{Q}_l}{\hat{P}_l} P_l \right) $. 
With this definition, 

\begin{align*}
& \left( \sum_l \sqrt{\frac{\hat{P}_l}{\hat{Q}_l} } Q_l \right)^{m_k} 
       \left( \sum_l \sqrt{\frac{\hat{Q_l}}{\hat{P}_l} } P_l \right)^{m_1} \\
&= \exp( - \hat{I} )^{\frac{m_k + m_1}{2}}  \left( \sum_l \sqrt{\frac{\hat{P}_l}{\hat{Q}_l}} Q_l \right)^{\frac{m_k - m_1}{2}} 
 \left( \sum_l \sqrt{\frac{\hat{Q}_l}{\hat{P}_l}} P_l \right)^{\frac{m_1 - m_k}{2}} 
\end{align*}

We claim that the following three statements are true. 
\begin{enumerate}
\item[Claim 1] $m_k \geq n_k -  \gamma n$ and likewise for $m_1$.
\item[Claim 2] $\hat{I} - I_L \geq - o(1) I_L$
\item[Claim 3] $\left( \sum_l \sqrt{\frac{\hat{P}_l}{\hat{Q}_l}} Q_l \right)^{\frac{m_k - m_1}{2}} 
 \left( \sum_l \sqrt{\frac{\hat{Q}_l}{\hat{P}_l}} P_l \right)^{\frac{m_1 - m_k}{2}} = \exp(\frac{n}{k}  o(I_L)  $
\end{enumerate}

Let us first suppose that these statements are true and see that term~\ref{eqn:Ihat_error_term} can be bounded. 


\begin{align*}
& \exp( - \hat{I} )^{\frac{m_1 + m_k}{2}}  \left( \sum_l \sqrt{\frac{\hat{P}_l}{\hat{Q}_l}} Q_l \right)^{\frac{m_k - m_1}{2}} 
 \left( \sum_l \sqrt{\frac{\hat{Q}_l}{\hat{P}_l}} P_l \right)^{\frac{m_1 - m_k}{2}}  \\
&\leq  \exp( - (I^* + (\hat{I} - I^*) )^{\frac{m_1 + m_k}{2}}  
 \left( \frac{\sum_l \sqrt{\frac{\hat{Q}_l}{\hat{P}_l}} P_l}
             {\sum_l \sqrt{\frac{\hat{P}_l}{\hat{Q}_l}} Q_l} \right)^{\frac{m_1 - m_k}{2}}  
  \\
&\leq \exp \left( - (1-o(1)) I_L (n_1 - \gamma n) \right) 
   \left( \frac{\sum_l \sqrt{\frac{\hat{Q}_l}{\hat{P}_l}} P_l}
             {\sum_l \sqrt{\frac{\hat{P}_l}{\hat{Q}_l}} Q_l} \right)^{\frac{m_1 - m_k}{2}}  
   \quad \trm{(by claim 1 and 2)}\\
&\leq \exp \left( - (1-o(1)) \frac{n}{\beta k} I_L  \right) 
   \quad \trm{(by claim 3)}
\end{align*}

The last inequality holds because $\gamma = o\left( \frac{1}{K \log K} \right)$. We will prove each of the three claims in the remainder of the proof.

\textbf{Claim 1:} This is straightforward. $\sigma_u$ has at most $\gamma n$ errors and therefore, $m_1' \geq n_1 - \gamma n$ and $m_1 - m_1' \leq \gamma n$. 

\textbf{Claim 2:} We show that the estimation error of $\hat{P}_l, \hat{Q}_l$ does not make $\hat{I}$ too small.

\begin{align}
\hat{I} - I_L &= - \log \frac{ 
     \left( \sum_l \sqrt{\frac{\hat{P}_l}{\hat{Q}_l}} Q_l \right)
     \left( \sum_l \sqrt{\frac{\hat{Q}_l}{\hat{P}_l}} P_l \right)}{ 
          \left( \sum_l \sqrt{P_l Q_l} \right)^2 } \label{eqn:Ihat_Istar}
\end{align}

Let us consider the numerator.
\begin{align*}
& \left( \sum_l \sqrt{ \frac{\hat{P}_l}{\hat{Q}_l}} Q_l \right)
\left( \sum_l \sqrt{ \frac{\hat{Q}_l}{\hat{P}_l}} P_l \right) \\
&= \left( \sum_l \sqrt{ P_l Q_l} \sqrt{ \frac{\hat{P}_l}{P_l} \frac{Q_l}{\hat{Q}_l}} \right) 
     \left( \sum_l \sqrt{P_l Q_l} \sqrt{ \frac{P_l}{\hat{P}_l} \frac{\hat{Q}_l}{ Q_l}} \right) \\
&= \sum_l P_l Q_l + 2\sum_{l < l'} \sqrt{P_l Q_l P_{l'} Q_{l'}} + 
   \sum_{l < l'} \sqrt{P_l Q_l P_{l'} Q_{l'}} \left( \sqrt{T_{l,l'}} + \frac{1}{\sqrt{T_{l,l'}}} - 2 \right) \\
&= \left( \sum_l \sqrt{P_l Q_l} \right)^2 + \sum_{l < l'} 
                  \sqrt{P_l Q_l P_{l'} Q_{l'}} \left( \sqrt{T_{l,l'}} + \frac{1}{\sqrt{T_{l,l'}}} - 2 \right) 
\end{align*}

where we define $T_{l,l'} = \frac{\hat{P}_l}{P_l} \frac{Q_l}{\hat{Q}_l} 
      \frac{P_{l'}}{\hat{P}_{l'}} \frac{\hat{Q}_{l'}}{Q_{l'}}  $. It will be later shown that $T_{l,l'} \rightarrow 1$ and thus, continuing equation~\ref{eqn:Ihat_Istar},

\begin{align}
\hat{I} - I_L &= - \log \left( 1 + \frac{ \sum_{l<l'} \sqrt{P_l Q_l P_{l'} Q_{l'}} 
    \left( \sqrt{T_{l,l'}} + \frac{1}{\sqrt{T_{l,l'}}} - 2 \right)}
    { \left( \sum_l \sqrt{ P_l Q_l} \right)^2 }  \right) \nonumber \\
     &  \geq  - \log \left( 1 + 4 \sum_{l<l'} \sqrt{P_l Q_l P_{l'} Q_{l'}}  
    \left( \sqrt{T_{l,l'}} + \frac{1}{\sqrt{T_{l,l'}}} - 2 \right)  \right) 
  \quad \trm{(assuming that $\sum_l \sqrt{P_l Q_l} \geq 1/2$)} \nonumber \\
   & \geq - 4 \sum_{l < l'} \sqrt{P_l Q_l P_{l'} Q_{l'}} 
    \left( \sqrt{T_{l,l'}} + \frac{1}{\sqrt{T_{l,l'}}} - 2 \right) \label{eqn:Ihat_Istar2}
\end{align}

We proceed by first bounding $|T_{l,l'} - 1|$ and then taking the second order approximation of $\left( \sqrt{T_{l,l'}} + \frac{1}{\sqrt{T_{l,l'}}} - 2 \right)$ around $1$. 

\begin{align*}
|T_{l,l'} - 1| &= \left| \frac{\hat{P}_l}{P_l} \frac{Q_l}{\hat{Q}_l} 
      \frac{P_{l'}}{\hat{P}_{l'}} \frac{\hat{Q}_{l'}}{Q_{l'}} - 1 \right| \\
 &= \left| \left( 1 - \frac{P_l - \hat{P}_l}{P_l} \right)
    \left( 1 - \frac{\hat{Q}_l - Q_l}{\hat{Q}_l} \right)
   \left( 1- \frac{\hat{P}_{l'} - P_{l'}}{\hat{P}_{l'}}\right)
   \left( 1 -  \frac{Q_{l'}- \hat{Q}_{l'}}{Q_{l'}} \right) -1 \right| \\
&\leq \left( \frac{|P_l - \hat{P}_l|}{P_l} +  \frac{|\hat{Q}_l - Q_l|}{\hat{Q}_l}
           +   \frac{| \hat{P}_{l'} - P_{l'}|}{\hat{P}_{l'}} +
               \frac{| Q_{l'} - \hat{Q}_{l'} | }{Q_{l'}} \right) \\
& \leq 2\left( \frac{|P_l - \hat{P}_l|}{P_l} +  \frac{|\hat{Q}_l - Q_l|}{Q_l}
           +   \frac{| \hat{P}_{l'} - P_{l'}|}{P_{l'}} +
               \frac{| Q_{l'} - \hat{Q}_{l'} | }{Q_{l'}} \right) 
\end{align*}
where the last inequality follows from lemma~\ref{lem:bound_ratio_P_Pl}.

Since we only work with pairs $(l, l')$ such that $l' > l$ and we can choose whatever ordering we would like. Suppose that the $l$'s are in decreasing order of $\frac{|\hat{P}_l - P_l|}{P_l} + \frac{|\hat{Q}_l - Q_l|}{Q_l}$ and therefore, we have that, for all pairs $l < l'$, 
\[
| T_{l,l'} - 1 | \leq 4
    \left( \frac{|\hat{P}_l - P_l|}{P_l} + \frac{|\hat{Q}_l - Q_l|}{Q_l} \right)
\]

By proposition~\ref{prop:estimation_consistency}, we have that, for $l \in L_1$, $\frac{|P_l - \hat{P}_l|}{P_l} \leq \eta \frac{\Delta_l}{P_l \vee Q_l}$ and for $l \notin L_1$, $\frac{|P_l - \hat{P}_l|}{P_l} \leq \eta \frac{1}{\sqrt{n (P_l \vee Q_l)}}$ and likewise for the $\frac{|\hat{Q}_l - Q_l|}{Q_l}$ term. We plug these bounds into the previous derivation and get that:
\begin{align*}
|T_{l,l'} - 1|  &\leq \eta'  \frac{\Delta_l}{P_l \vee Q_l}  \quad \trm{for $l \in L_1$}\\
|T_{l,l'} - 1 | &\leq \eta' \frac{1}{\sqrt{ n (P_l \vee Q_l)}} \quad \trm{for $l \notin L_1$}
\end{align*}
Where we have used the assumption that $\frac{1}{\rho_L} \leq \frac{P_l}{Q_l} \leq \rho_L$ and $\eta \rho_L \rightarrow 0$. 


The Taylor approximation of $\sqrt{T_{l,l'}} + \frac{1}{\sqrt{T_{l,l'}}} - 2$ around $T_{l,l'}=1$ is:
\begin{align*}
\sqrt{T_{l,l'}} + \frac{1}{\sqrt{T_{l,l'}}} -2  &\leq 
  \frac{1}{4} (T_{l,l'} - 1)^2 + O (T_{l,l'}-1)^3 
\end{align*}

Continuing on from equation~\ref{eqn:Ihat_Istar2}, we have that
\begin{align*}
\hat{I} - I_L &\geq - 4 \sum_{l < l'} \sqrt{P_l Q_l P_{l'} Q_{l'}} 
    \left( \sqrt{T_{l,l'}} + \frac{1}{\sqrt{T_{l,l'}}} - 2 \right) \\
&\geq - 4 \sum_{l \in L_1} \sum_{l' > l} \sqrt{P_l Q_l P_{l'} Q_{l'}} 
    \left( \sqrt{T_{l,l'}} + \frac{1}{\sqrt{T_{l,l'}}} - 2 \right) 
     - 4 \sum_{l \notin L_1} \sum_{l' > l} \sqrt{P_l Q_l P_{l'} Q_{l'}} 
    \left( \sqrt{T_{l,l'}} + \frac{1}{\sqrt{T_{l,l'}}} - 2 \right) \\
  &\geq - \sum_{l \in L_1} \sum_{l' > l} \sqrt{P_l Q_l P_{l'} Q_{l'}} 
             \eta' \left( \frac{\Delta_l}{P_l \vee Q_l}  \right)^2 
        - \sum_{l \notin L_1} \sum_{l' > l} \sqrt{P_l Q_l P_{l'} Q_{l'}} 
             \eta' \frac{1}{n (P_l \vee Q_l)} \\
 &\geq - \eta' \left( \sum_{l \in L_1} \frac{\Delta_l^2}{P_l \vee Q_l} \right)
         \left( \sum_{l'}  \sqrt{P_{l'}Q_{l'}} \right) 
       - \eta' \left( \sum_{l \notin L_1} \frac{1}{n} \right) 
          \left( \sum_{l'} \sqrt{P_{l'} Q_{l'} } \right) \\
 &\geq - o(I_L)
\end{align*}

The last inequality follows because $\sum_{l'} \sqrt{P_{l'} Q_{l'}} \leq 1$ and because $\sum_{l \notin L_1} \frac{1}{n} \leq \frac{L}{n} = o(I_L)$. This proves claim 2.

\textbf{Claim 3.} 

\begin{align*}
& \left( \sum_l \sqrt{\frac{\hat{P}_l}{\hat{Q}_l}} Q_l \right)^{\frac{m_k - m_1}{2}} 
 \left( \sum_l \sqrt{\frac{\hat{Q}_l}{\hat{P}_l}} P_l \right)^{\frac{m_1 - m_k}{2}} \\
&= \left( \sum_l \sqrt{\frac{\hat{P}_l}{\hat{Q}_l}} Q_l \right)^{\frac{m_k - m_1}{2}} 
 \left( \sum_l \sqrt{\frac{\hat{Q}_l}{\hat{P}_l}} P_l \right)^{\frac{m_1 - m_k}{2}} 
   \left( \frac{\sum_l \sqrt{\hat{P}_l \hat{Q}_l}}{\sum_l \sqrt{\hat{P}_l \hat{Q}_l}} \right)^{\frac{m_1 - m_k}{2}} \\
&=  \left( 
   \frac{\sum_l \sqrt{\frac{\hat{P}_l}{\hat{Q}_l}} Q_l}
        {\sum_l \sqrt{\frac{\hat{P}_l}{\hat{Q}_l}} \hat{Q}_l} 
     \right)^{\frac{m_k - m_1}{2}} 
   \left( \frac{\sum_l \sqrt{\frac{\hat{Q}_l}{\hat{P}_l}} P_l}
         {\sum_l \sqrt{\frac{\hat{Q}_l}{\hat{P}_l}} \hat{P_l} } \right)^{\frac{m_1 - m_k}{2}} 
\end{align*}

Assume that $m_k \geq m_1$. The reverse case can be analyzed in the identical manner. Then,
\begin{align*}
&= \left( 1 + 
   \frac{\sum_l \sqrt{\frac{\hat{P}_l}{\hat{Q}_l}} (Q_l - \hat{Q}_l)}
        {\sum_l \sqrt{\frac{\hat{P}_l}{\hat{Q}_l}} \hat{Q}_l} 
     \right)^{\frac{m_k - m_1}{2}} 
   \left( 1+ \frac{\sum_l \sqrt{\frac{\hat{Q}_l}{\hat{P}_l}} (\hat{P}_l - P_l)}
         {\sum_l \sqrt{\frac{\hat{Q}_l}{\hat{P}_l}} P_l } \right)^{\frac{m_k - m_1}{2}} 
   \\
\end{align*}

By lemma~\ref{lem:sqrt_ratio_times_ql}, the denominators are of constant order. That is, 
$\sum_l \sqrt{ \frac{ \hat{P}_l }{ \hat{Q}_l } \hat{Q}_l } = C$ and 
$\sum_l \sqrt{ \frac{\hat{Q}_l}{P_l} } P_l = C$. 

To bound the numerator term, we apply lemma~\ref{lem:sqrt_ratio_pl_ql_minus_1}. 

\begin{align*}
\left| \sum_l \sqrt{\frac{\hat{P}_l}{\hat{Q}_l}} (Q_l - \hat{Q}_l) \right|  &= 
  \left|  \sum_l \left( \sqrt{\frac{\hat{P}_l}{\hat{Q}_l}} -1 \right) (Q_l - \hat{Q}_l) 
 \right| \\
& \leq \left| \sum_{l \in L_1} \left( \sqrt{\frac{\hat{P}_l}{\hat{Q}_l}} -1 \right) (Q_l - \hat{Q}_l) \right| +  %next term
  \left| \sum_{l \notin L_1} \left( \sqrt{\frac{\hat{P}_l}{\hat{Q}_l}} -1 \right) (Q_l - \hat{Q}_l) \right| \\
& \leq \sum_{l \in L_1} \eta' \frac{\Delta_l^2}{Q_l} + \sum_{l \notin L_1} \eta' \frac{1}{n} \\
& \leq \eta' I_L + \eta' \frac{L}{n}  \\
& \leq \eta' I_L  
\end{align*}

The second inequality follows from lemma~\ref{lem:sqrt_ratio_pl_ql_minus_1} and the definition of $L_1$. 

\begin{align*}
& \left( 1 + 
   \frac{\sum_l \sqrt{\frac{\hat{P}_l}{\hat{Q}_l}} (Q_l - \hat{Q}_l)}
        {\sum_l \sqrt{\frac{\hat{P}_l}{\hat{Q}_l}} \hat{Q}_l} 
     \right)^{\frac{m_k - m_1}{2}} 
   \left( 1+ \frac{\sum_l \sqrt{\frac{\hat{Q}_l}{\hat{P}_l}} (\hat{P}_l - P_l)}
         {\sum_l \sqrt{\frac{\hat{Q}_l}{\hat{P}_l}} P_l } \right)^{\frac{m_k - m_1}{2}} 
\\
&\leq \exp\left( (m_k - m_1) \log(1 + o(I_L) ) \right) \\
&\leq \exp \left( \frac{n}{K} o(I_L) \right) 
\end{align*}

This proves claim 3. 

Multiplying the bounds for term~\ref{eqn:Ihat_error_term} and \ref{eqn:excess_error_term} shows that the probability of misclustering $u$ into some cluster $k \neq 1$ is at most $\exp\left(( - (1 + o(1)) \frac{nI_L}{\beta K} \right)$. Taking a union bound over all clusters $k \neq $ completes the proof.


\end{proof}



\subsubsection{Lemmas}


Here we collect various lemmas used in the proof.

We often use the bound that $\frac{1}{2} P \leq \hat{P}_l \leq 2 P_l$. The following lemma justifies this.

\begin{lemma}
\label{lem:bound_ratio_P_Pl}
Let $l$ be any color and suppose that $\frac{1}{\rho_L} \leq \frac{P_l}{Q_l} \leq \rho_L$ where $\rho_L > 1$; suppose also that $P_l, Q_l \geq \frac{c}{n}$ for some absolute constant $c$.

Let us condition on the event that the 
conclusion of proposition~\ref{prop:estimation_consistency} holds with a sequence $\eta$ such that $\eta \rho_L \rightarrow 0$. 

Then we have that
\[
\max_l \frac{|\hat{P}_l - P_l|}{P_l} \rightarrow 0
\quad \trm{ and } \quad
\max_l \frac{|\hat{Q}_l - Q_l|}{Q_l} \rightarrow 0
\]

In particular, for small enough $\eta$, we have that $\frac{1}{2} P \leq \hat{P}_l \leq 2P_l$ for all $l$ and likewise for $Q_l$.

\end{lemma}


\begin{proof}

We prove the statement first for $P_l$; the same argument goes for $Q_l$. 

By proposition~\ref{prop:estimation_consistency}, we have that either $| \hat{P}_l - P_l | \leq \eta \Delta_l$ or $| \hat{P}_l - P_l | \leq \eta \sqrt{ \frac{P_l \vee Q_l}{n} }$.

Let us suppose $| \hat{P}_l - P_l | \leq \eta \Delta_l$ first. Then,

\begin{align*}
\frac{ | \hat{P}_l - P_l |}{P_l} \leq \eta \frac{\Delta_l}{P_l} \leq \eta \rho_L \rightarrow 0
\end{align*}

Now suppose that $| \hat{P}_l - P_l | \leq \eta \sqrt{ \frac{P_l \vee Q_l}{n} }$. Then,

\begin{align*}
\frac{| \hat{P}_l - P_l |}{P_l} \leq \eta \sqrt{ \frac{\rho_L}{ P_l n}} \leq \eta \rho_L \sqrt{ \frac{1}{c}} \rightarrow 0
\end{align*}

\end{proof}



\begin{lemma}
\label{lem:sqrt_ratio_pl_ql_minus_1}
Suppose that $\frac{1}{\rho_L} \leq \frac{P_l}{Q_l} \leq \rho_L$ for $\rho_L > 1$ and that $P_l, Q_l \geq \frac{c}{n}$ for some absolute constant $c$.\\ 

Let us condition on the event that the 
conclusion of proposition~\ref{prop:estimation_consistency} holds with a sequence $\eta$ such that $\eta \rho_L \rightarrow 0$. 

Let $C_{thresh}$ be the constant defined in proposition~\ref{prop:estimation_consistency}. Then, the following are true:
\begin{enumerate}
\item 
Suppose $l$ satisfies $n \frac{\Delta_l^2}{P_l \vee Q_l} \geq C_{thresh}$. Then, we have that 
\[
\left| \sqrt{ \frac{\hat{P}_l }{\hat{Q}_l} } - 1 \right|
                    \leq \left| \frac{P_l - Q_l}{Q_l} \right| (1 + \eta') 
\]
where $\eta' \rightarrow 0$ and does not depend on the color $l$. 

\item
Suppose $l$ satisfies $n \frac{\Delta_l^2}{P_l \vee Q_l} \leq C_{thresh}$. Then, we have that
\[
\left| \sqrt{ \frac{\hat{P}_l}{\hat{Q}_l} } - 1 \right| \leq
 2 C_{thresh} \frac{1}{\sqrt{n  (P_l \vee Q_l) } }
\]

\end{enumerate}
And the same conclusion follows for $\sqrt{ \frac{\hat{Q}_l}{\hat{P}_l}} - 1$.

\end{lemma}



\begin{proof}
First, suppose that $l$ satisfies $n \frac{\Delta_l^2}{P_l \vee Q_l} \geq C_{thresh}$. 

Note that 
\begin{align*}
 \frac{P_l}{Q_l} - 1 &= \frac{P_l - Q_l}{Q_l}  
\end{align*}

We will show that $\frac{\hat{P}}{\hat{Q}}$ behaves similarly. 

As a preliminary step, we note that, by lemma~\ref{lem:bound_ratio_P_Pl}, $ \frac{\hat{Q}_l - Q_l}{Q_l} = \eta'$ where $\max_l |\eta'| \rightarrow 0$.

In the following derivation, we use $\eta'$ to denote a sequence such that $\max_l |\eta'| = o(1)$; the actual value of $\eta'$ may change from instance to instance. 
We use $\eta$ to denote a sequence where $\max_l |\eta \rho_L| = o(1)$.

\begin{align*}
\frac{\hat{P}_l}{\hat{Q}_l} - 1 &= 
     \frac{ \hat{P}_l - P_l + P_l }{ \hat{Q}_l - Q_l + Q_l} -1  \\
  &=  \frac{  \frac{\hat{P}_l - P_l}{Q_l} + \frac{P_l}{Q_l}}
       { \frac{\hat{Q}_l - Q_l}{Q_l} + 1} - 1 \\
 &= \left( \frac{P_l}{Q_l} + \frac{\hat{P}_l - P_l}{Q_l} \right)
    \left( 1 - \frac{\hat{Q}_l - Q_l}{Q_l} (1 + \eta')  \right) -1  \\
 &= \frac{P_l}{Q_l} + \frac{\hat{P}_l - P_l}{Q_l} 
     - \frac{P_l}{Q_l} \frac{\hat{Q}_l - Q_l}{Q_l} (1 + \eta') 
     - \frac{\hat{P}_l - P_l}{Q_l} \frac{\hat{Q}_l - Q_l}{Q_l}(1+ \eta') 
   - 1  \\
 &= \frac{P_l}{Q_l} + \eta \frac{\Delta_l}{Q_l}
     + \eta \rho_L \frac{\Delta_l}{Q_l}
   - 1  \\
 &= \frac{P_l - Q_l}{Q_l} ( 1 + \eta')
\end{align*}
For the second to last equality, we used the fact that $| \hat{P}_l - P_l| \leq \eta \Delta_l$ by the conclusion of proposition~\ref{prop:estimation_consistency}.

If $P_l \geq Q_l$, then, for small enough value of $\eta'$, we have that $\frac{\hat{P}_l}{\hat{Q}_l} \geq 1$. Hence,

\begin{align*}
\sqrt{ \frac{\hat{P}_l}{\hat{Q}_l} } - 1 &\leq \frac{\hat{P}_l}{\hat{Q}_l} - 1 \\
   &\leq \frac{P_l - Q_l}{Q_l} (1 + \eta')
\end{align*}
If $P_l \leq Q_l$, then, for small enough value of $\eta'$, we have that $\frac{\hat{P}_l}{\hat{Q}_l} \leq 1$. Hence,

\begin{align*}
\sqrt{ \frac{\hat{P}_l}{\hat{Q}_l} } - 1 &\geq \frac{\hat{P}_l}{\hat{Q}_l} - 1 \\
   &\geq \frac{P_l - Q_l}{Q_l} (1 + \eta')
\end{align*}




Symmetry yields that 

\begin{align*}
\sqrt{ \frac{\hat{Q}_l}{\hat{P}_l} } - 1  &
   \left\{ \begin{array}{cc}
      \leq \frac{Q_l - P_l}{P_l} (1 + \eta') & \trm{(if $Q_l \geq P_l$)} \\
      \geq \frac{Q_l - P_l}{P_l} (1 + \eta') & \trm{(if $Q_l \leq P_l$)} 
     \end{array} \right. \\
\end{align*}

This proves the first case. \\

The proof of the second case is almost identical. Let us assume that $l$ is such that $n \frac{\Delta_l^2}{P_l \vee Q_l} \leq C_{thresh}$. In this case, we have that

\begin{align*}
| \hat{P}_l - P_l | &= \eta \sqrt{ \frac{P_l \vee Q_l}{n} } \\
| \hat{Q}_l - Q_l | &= \eta \sqrt{ \frac{P_l \vee Q_l}{n} }
\end{align*}

where $\max_l |\eta \rho_L| \rightarrow 0$. 

By lemma~\ref{lem:bound_ratio_P_Pl}, $\frac{\hat{Q}_l - Q}{Q_l} = \eta'$ where $\max_l |\eta'| = o(1)$.

In the following derivation, we use $\eta'$ to denote a sequence such that $\max_l |\eta'| = o(1)$; the actual value of $\eta'$ may change from instance to instance. 
\begin{align*}
\frac{\hat{P}_l}{\hat{Q}_l} - 1 &= 
     \left( \frac{P_l}{Q_l} + \frac{\hat{P}_l - P_l}{Q_l} \right)
     \left( 1 - \frac{\hat{Q}_l - Q_l}{Q_l} ( 1 + \eta') \right) - 1 \\
 &= \frac{P_l}{Q_l} + \frac{\hat{P}_l - P_l}{Q_l} 
     - \frac{P_l}{Q_l} \frac{\hat{Q}_l - Q_l}{Q_l} (1 + \eta') 
     - \frac{\hat{P}_l - P_l}{Q_l} \frac{\hat{Q}_l - Q_l}{Q_l}(1+ \eta') 
   - 1  \\
 &= \frac{P_l - Q_l}{Q_l} + \eta' \sqrt{ \frac{1}{ n (P_l \vee Q_l)} } 
\end{align*}

and, because $\frac{\hat{P}_l}{\hat{Q}_l} > 0$,  it is clear that $\eta'$ satisfies the condition that $\frac{P_l - Q_l}{Q_l} + \eta'\sqrt{ \frac{1}{n (P_l \vee Q_l)} } + 1 > 0$. 



\begin{align*}
\left| \sqrt{ \frac{\hat{P}_l}{\hat{Q}_l} } - 1 \right| &= 
 \left|  \sqrt{ 1 + \frac{P_l - Q_l}{Q_l} + \eta' \frac{1}{\sqrt{n (P_l \vee Q_l)} }}
   -1  \right| \\
  &\leq \left| \frac{P_l - Q_l}{Q_l} + \eta' \frac{1}{\sqrt{ n (P_l \vee Q_l)}}        \right| \\
 &\leq 2 C_{thresh} \frac{1}{\sqrt{n (P_l \vee Q_l)} } 
\end{align*}

The first inequality follows because $ \sqrt{1 + x} - 1 \leq x$ for $x \geq 0$ and $\sqrt{ 1 + x} - 1 \geq x$ for $-1 < x < 0$. 
The second inequality follows because $\left| \frac{P_l - Q_l}{Q_l} \right| \leq C_{thresh} \frac{1}{\sqrt{n (P_l \vee Q_l)}} $. 

\end{proof}




The following lemma bounds $\sqrt{ \frac{\hat{P}_l}{\hat{Q}_l} } Q_l$. 
\begin{lemma}
\label{lem:sqrt_ratio_times_ql}
Suppose that
\[
\frac{|\hat{Q}_l - Q_l|}{Q_l} = \eta' \quad 
\frac{|\hat{P}_l - P_l|}{P_l} = \eta'
\]
where $\max_l |\eta'| = o(1)$. 


Then, we have that for all small enough $\eta$, 
\[
\sqrt{ \frac{\hat{P}_l}{\hat{Q}_l} } Q_l \geq \frac{1}{2} \sqrt{\hat{P}_l \hat{Q}_l} \geq
  \frac{1}{8} \sqrt{P_l Q_l}
\]
\end{lemma}


\begin{proof}

\begin{align*}
& \sqrt{\frac{\hat{P}_l}{\hat{Q}_l} } Q_l \\
&= \sqrt{ \hat{P}_l \hat{Q}_l} \frac{Q_l}{\hat{Q}_l} \\
&= \sqrt{\hat{P}_l \hat{Q}_l} \frac{1}{ \frac{Q_l - \hat{Q}_l}{Q_l} + 1 } \\
&= \sqrt{\hat{P}_l \hat{Q}_l} 
  \left( 1 - \frac{\hat{Q}_l - Q_l}{Q_l} (1 + \eta') \right)  \\
&= \sqrt{\hat{P}_l \hat{Q}_l} (1 - \eta)
\end{align*}

where in the second to last equality, we used the fact that $\frac{\hat{Q}_l - Q_l}{Q_l} = \eta' \rightarrow 0$. 

Clearly then, for small enough $\eta$, we continue the bound as 
\begin{align*}
&\geq \frac{1}{2} \sqrt{ \hat{P}_l \hat{Q}_l} 
\end{align*}

Also, for small enough $\eta$, we have that $\hat{P}_l \geq \frac{1}{2} P_l$ and $\hat{Q}_l \geq \frac{1}{2} Q_l$ and thus, we have the final bound
\[
\frac{1}{8} \sqrt{ P_l Q_l}
\]
as desired. 

\end{proof}






We state Lemma 4 from~\cite{gao2015achieving} which analyzes the consensus step of the algorithm.

\begin{lemma} 
\label{lem:consensus_analysis}
Let $\sigma, \sigma'$ be two clusters such that, for some constant $C \geq 1$, the minimum cluster size is at least $ \frac{n}{Ck}$. \\

Define a map $\xi \,:\, [k] \rightarrow [k]$ as $\xi(k) = \argmax_{k'} | \{ v \,:\, \sigma(v) = k \} \cap \{ v \,:\, \sigma'(v) = k' \} |$.\\

Then, if $\min_{\pi \in S_k} l(\pi(\sigma), \sigma') < \frac{1}{Ck}$, we have that $\xi \in S_k$ and $l(\xi(\sigma), \sigma') = \min_{\pi \in S_k} l(\pi(\sigma), \sigma')$. 
\end{lemma}

We include a simple additional lemma.
\begin{lemma}
\label{lem:consensus_uniqueness}
Let $\sigma, \sigma' \,:\, [n] \rightarrow [k]$ be two clusterings where the minimum cluster size of $\sigma$ is $T$. Let $\pi, \xi \in S_k$ be such that
\[
d(\pi(\sigma), \sigma') < T/2 \qquad \qquad d(\xi(\sigma), \sigma') < T/2
\]
Then it must be that $\pi = \xi$.
\end{lemma}

\begin{proof}
Suppose not, then choose any $k$ such that $\pi(k) \neq \xi(k)$. 

\begin{align*}
| \{ \sigma(u) = k \} \cap \{ \sigma'(u) \neq \pi(k) \} | < d(\pi(\sigma), \sigma') < T/2
\end{align*}

So, then, we have that $| \{ \sigma(u) = k \} \cap \{ \sigma'(u) = \pi(k) \}| > T/2$. 

But then,
\begin{align*}
d(\xi(\sigma), \sigma') &\geq | \{ \sigma(u) = k \} \cap \{\sigma'(u) \neq \xi(k) \} | \\
   &\geq | \{ \sigma(u) = k\} \cap \{ \sigma'(u) = \pi(k) \}| \\ 
   &\geq T/2
\end{align*}

\end{proof}





%%% Local Variables:
%%% mode: latex
%%% TeX-master: "../paper_all_in_one"
%%% End:

