

\section{Proof of Proposition~\ref{prop:theta_rate}}
\label{sec:theta_rate_proof}

%We first review the setting of proposition~\ref{prop:theta_rate}.

%Let $p(x)$ and $q(x)$ be two members of a family of distributions parametrized by $\theta$. We write

%\[
%p(x) = \exp( f_{\theta_1}(x)) \quad q(x) = \exp(f_{\theta_0}(x))
%\]

%We restrict $\theta_1, \theta_0$ to be in some compact set $\Theta \subset \R^{d_\Theta}$. 

%Define the Fisher Information
%\[
%G_{\theta} = \int_{-\infty}^{\infty} (\nabla_{\theta} f_{\theta}(x)) 
%                                    (\nabla_{\theta} f_{\theta}(x))^\tran 
%                      \exp( f_{\theta}(x)) dx
%\]

%\begin{enumerate}
%\item[B1] For all $\theta \in \Theta$, $\liminf_{|x| \rightarrow \infty} \left( g(x) - f_\theta(x) \right) > -\infty$. 
%\item[B2] We require that Fisher Information is full-ranked.
%   \[
% 0<   c_{\min} <  \inf_{\theta} \lambda_{min}(G_\theta) \leq \sup_{\theta} \lambda_{max}(G_{\theta}) < c_{\max} < \infty
%  \]
%\item[B3] There is some constant $c > 0$ such that, $\sup_\theta$ $\| \nabla_{\theta} f_{\theta} (x) \|$ is monotonically non-decreasing in $|x|$ for $|x| \geq c$.
%\item[B4] The following integrability condition holds.
%  \[
%   \int_{-\infty}^\infty \sup_{\theta} \| \nabla_{\theta} f_\theta(x) \|^{r \vee \frac{2t}{1-t}} \phi(x) dx < \infty
%  \] 
%  \[
%  \sup_{\theta} \int_{-\infty}^\infty \| \nabla_{\theta} f'_\theta(x) \|^{2t/(1-t)} \phi(x) dx < \infty
%  \]
%  \begin{align*}
%  \sup_{\theta} \int_{-\infty}^\infty | f'_{\theta}(x) |^{2t/(1-t)} \phi(x) &< \infty   \\
%        \int_{-\infty}^\infty | g'(x) |^{2t/(1-t)} \phi(x) &< \infty   
%  \end{align*}
%\item[B5] There is some constant $c' > 0$ such that, for all $\theta$, $f'_\theta(x) \geq g'(x) > 0$ for all $x \leq -c'$ and 
%          $f'_\theta(x) \leq g'(x) < 0$ for all $x \geq c'$. 
%\end{enumerate}

%%%%%

First suppose $\| \theta_1 - \theta_0 \| \rightarrow 0$. Then $\int (\sqrt{p(x)} - \sqrt{q(x)})^2 dx \rightarrow 0$ by Lemma~\ref{lem:hellinger_theta_equivalence}. Note that condition A1 follows directly from condition B1 and condition A5 follows directly from condition B5. In Propositions~\ref{prop:interval_existence}, \ref{prop:theta_A3_bound}, ~\ref{prop:theta_A4_bound} below, we establish conditons A2, A3, and A4, respectively.

Now suppose that $\| \theta_1 - \theta_0 \| = \Theta(1)$. Then $\int (\sqrt{p(x)} - \sqrt{q(x)})^2 dx = \Theta(1)$ by Lemma~\ref{lem:hellinger_theta_equivalence}. Condition A1' follows directdly from B1, while condition A4' follows directly from condition B5. In Propositions~\ref{prop:interval_existence} with $\rho = \exp(L^{1/r})$, we derive condition A2'. In Proposition~\ref{prop:theta_A4_bound}, we derive condition A3'.

\subsection{Supporting propositions}

\begin{proposition}
\label{prop:interval_existence}
Suppose assumption B3, B4 holds. There exists an interval $R \subset \{ x \,:\, \frac{1}{\rho} \leq \left| \frac{p(x)}{q(x)} \right| \leq \rho \}$ such that
\[
\Phi \{ R^c \} \leq C \frac{ \| \theta_1 - \theta_0 \|^r }{ (\log \rho)^r } \leq \frac{H^r}{ (\log \rho)^r}
\]
where $H \equiv \int (\sqrt{p(x)} - \sqrt{q(x)} )^2 dx$. 
\end{proposition}

\begin{proof}
We start with the observation that
\begin{align*}
\log \frac{p(x)}{q(x)} &= f_{\theta_1}(x) - f_{\theta_0}(x)\\
           &= (\theta_1 - \theta_0)^\tran \nabla_{\theta} f_{\bar{\theta}} (x) \\
| \log \frac{p(x)}{q(x)} | &\leq \| \theta_1 - \theta_0 \| \| \nabla_{\theta} f_{\bar{\theta}} (x) \| \\
                &\leq  \| \theta_1 - \theta_0 \| \sup_\theta \| \nabla_{\theta} f_{\theta} (x) \|
\end{align*}
where $\bar{\theta}$ is some convex combination of $\theta_0, \theta_1$. 

Therefore,  
$ \{ x \,:\, \frac{1}{\rho} \leq \left| \frac{p(x)}{q(x)} \right| \leq \rho \} 
         \supset
   \left\{ x \,:\, \sup_\theta \| \nabla_{\theta} f_{\theta} (x) \| \leq \frac{\log \rho}{\| \theta_1 - \theta_0 \|} \right\}$ and we take the latter quantity to be $R$. 

$R$ is an interval by B3 for small enough $\| \theta_1 - \theta_0 \|$ or large enough $\rho$.

Since we have that

\[
 \int_{-\infty}^{\infty} \sup_\theta \| \nabla_{\theta} f_{\theta}(x) \|^{r} \phi(x) dx < \infty
\]


By Markov's inequality, we have

\[
\Phi \left\{ x \,:\, \sup_\theta \| \nabla_{\theta} f_{\theta}(x) \|  > \frac{\log \rho}{\| \theta_1 - \theta_0 \|} \right\}
    \leq C \frac{\| \theta_1 - \theta_0 \|^{r}}{(\log \rho)^{r}}
\]

An application of lemma~\ref{lem:hellinger_theta_equivalence} finishes the proof.

\end{proof}




\begin{proposition}
\label{prop:theta_A3_bound}
Suppose assumptions B1-B4 are satisfied. Let $R \subset \R$ be the interval in proposition~\ref{prop:interval_existence}.

Then, we have that,

\[
\int_R \frac{1}{\alpha} \left| \frac{p(x)}{q(x)} - 1 \right|^r q(x) dx \leq \infty
\]

\end{proposition}

\begin{proof}

By lemma~\ref{lem:chi_square_theta_equivalence}, we have that $\alpha \asymp \| \theta_1 - \theta_0 \|$. Thus, we need only show that
\[
\int_R \frac{1}{\| \theta_0 - \theta_1 \|} \left| \frac{p(x)}{q(x)} - 1 \right|^r q(x) dx \leq \infty
\]




\begin{align*}
\frac{1}{\| \theta_1 - \theta_0\|}  \left| \frac{p(x)}{q(x)} - 1 \right| &=
    \frac{1}{\| \theta_1 - \theta_0\|}  
        \left| \exp( f_{\theta_1}(x) - f_{\theta_0}(x) ) - 1 \right| \\
   &=  \frac{1}{\| \theta_1 - \theta_0\|} \Big|(\theta_1 - \theta_0)^\tran \nabla_{\theta} f_{\bar{\theta}}(x)  \Big|
      \exp( f_{\bar{\theta}}(x) - f_{\theta_0}(x)) \\
  &\leq \| \nabla_{\theta} f_{\bar{\theta}}(x) \|   \exp( f_{\bar{\theta}}(x) - f_{\theta_0}(x)) \\
  &= \|  \nabla_{\theta} f_{\bar{\theta}}(x) \| 
       \exp\Big( (\bar{\theta} - \theta_1)^\tran \nabla_{\theta} f_{\tilde{\theta}}(x) \Big) \\
  &\leq  \|  \nabla_{\theta} f_{\bar{\theta}}(x) \| 
        \exp\Big( \| \theta_1 - \theta_0\| \| \nabla_{\theta} f_{\tilde{\theta}}(x) \| \Big)
\end{align*}

Where $\bar{\theta}, \tilde{\theta}$ are some convex combinations of $\theta_0, \theta_1$.

Therefore, 

\begin{align*}
\int_R \left( \frac{1}{\| \theta_1 - \theta_0\|} \left| \frac{p(x)}{q(x)} - 1 \right| \right)^r q(x) dx &\leq
    \int_R  \|  \nabla_{\theta} f_{\bar{\theta}}(x) \|^r
        \exp\Big( r \| \theta_1 - \theta_0\| \| \nabla_{\theta} f_{\tilde{\theta}}(x) \| \Big)  q(x) dx \\
  &\leq   \int_R  \|  \nabla_{\theta} f_{\bar{\theta}}(x) \|^r
        e^{r \log \rho}  q(x) dx \\
  &= \int_{R}  \|  \nabla_{\theta} f_{\bar{\theta}}(x) \|^r
       \rho^r  q(x) d x \\
  &\leq \rho^r  \int_{-\infty}^{\infty} \|  \nabla_{\theta} f_{\bar{\theta}}(x) \|^r
        q(x) d x  < \infty
\end{align*}



\end{proof}




\begin{proposition}
\label{prop:theta_A4_bound}
Suppose assumptions B1-B4 are satisfied. Let $R \subset \R$ be the interval in proposition~\ref{prop:interval_existence}.

Then, we have that,

\[
\int_R \left| \frac{1}{\alpha} \frac{p'(x) - q'(x)}{q(x)} \right|^{t/(1-t)} \phi(x) dx < \infty
\]

\end{proposition}

\begin{proof}

Again, using the fact that $\alpha \asymp \| \theta_1 - \theta_0 \|$, we need only prove that
\[
\int_R \left| \frac{1}{\| \theta_1 - \theta_0\| } \frac{p'(x) - q'(x)}{q(x)} \right|^{t/(1-t)} \phi(x) dx < \infty
\]


Note that
\begin{align*}
\frac{1}{\| \theta_0 - \theta_1 \|} \frac{p'(x) - q'(x)}{q(x)}& = 
   \frac{1}{\| \theta_0 - \theta_1 \|} \left[
           f'_{\theta_1} \frac{p(x)}{q(x)} - f_{\theta_0}'(x) \right] \\
  &= \frac{1}{\|\theta_0 - \theta_1\|} \left\{ 
          ( f'_{\theta_1}(x) - f'_{\theta_0}(x) ) \frac{p(x)}{q(x)} 
       + f'_{\theta_0} \left( \frac{p(x)}{q(x)} - 1 \right) \right\} \\
  &= \nabla_{\theta} f'_{\bar{\theta}}(x) \frac{p(x)}{q(x)} 
         + \frac{1}{\| \theta_1 - \theta_0\|} f'_{\theta_0}(x) \left( \frac{p(x)}{q(x)} - 1 \right) \\
\end{align*}
where $\bar{\theta}$ is some convex combination of $\theta_1, \theta_0$. 

Therefore, we have:
\begin{align*}
& \int_R \left| \frac{1}{\| \theta_1 - \theta_0\| } \frac{p'(x) - q'(x)}{q(x)} \right|^{t/(1-t)} \phi(x) dx \\
& =  \int_R \left|  \nabla_{\theta} f'_{\bar{\theta}}(x) \frac{p(x)}{q(x)} 
         + \frac{1}{\| \theta_1 - \theta_0\|} f'_{\theta_0}(x) \left( \frac{p(x)}{q(x)} - 1 \right)  \right|^{t/(1-t)} \phi(x)dx
\end{align*}


To show that this integral is finite, we need only show, regardless of the value of $t/(1-t)$, that the two components have finite integrals:

\[
 \int_R \left|  \nabla_{\theta} f'_{\bar{\theta}}(x) \frac{p(x)}{q(x)} \right|^{t/(1-t)} \phi(x) dx \quad \trm{ and } \quad
         \int_R \left| \frac{1}{\| \theta_1 - \theta_0\|} f'_{\theta_0}(x) \left( \frac{p(x)}{q(x)} - 1 \right)  \right|^{t/(1-t)} \phi(x)dx
\]


We bound the first integral.

\begin{align*}
 \int_R \left|  \nabla_{\theta} f'_{\bar{\theta}}(x) \frac{p(x)}{q(x)} \right|^{t/(1-t)} \phi(x) dx &\leq 
            \int_R \left|  \nabla_{\theta} f'_{\bar{\theta}}(x) \right|^{t/(1-t)} \rho \phi(x) dx  < \infty
\end{align*}
         
And then the second.
\begin{align*}
&  \int_R \left| \frac{1}{\| \theta_1 - \theta_0\|} f'_{\theta_0}(x) \left( \frac{p(x)}{q(x)} - 1 \right)  \right|^{t/(1-t)} \phi(x)dx \\
    &\leq \int_R | f'_{\theta_0}(x) | 
             \left|  \frac{1}{\| \theta_1 - \theta_0\|}  \left( \frac{p(x)}{q(x)} - 1 \right)  \right|^{t/(1-t)} \phi(x) dx \\
    &\leq \left\{ \int_R  | f'_{\theta_0}(x) |^{2t/(1-t)} \phi(x) dx \right\}^{1/2} 
         \left\{
            \int_R
             \left| \frac{1}{\| \theta_1 - \theta_0\|}  \left( \frac{p(x)}{q(x)} - 1 \right)  \right|^{2t/(1-t)} \phi(x) dx 
         \right\}^{1/2} 
\end{align*}

The first quantity is finite by assumption. A bound for the second quantity follows from proposition~\ref{prop:theta_A3_bound}.

\end{proof}



\subsection{Supporting lemmas}

\begin{lemma}
\label{lem:chi_square_theta_equivalence}
Suppose assumptions B1-B4 hold.
Let $R$ be the interval in Proposition~\ref{prop:interval_existence}. Define $\alpha = \int_R q(x) \left( \frac{p(x)}{q(x)} - 1 \right)^2 dx$. Then we have that
\[
\alpha \asymp \| \theta_1 - \theta_0 \|
\]

\end{lemma}

\begin{proof}

\begin{align*}
\alpha^2 &= \int_R \left( \frac{p(x)}{q(x)} - 1 \right)^2 q(x) dx \\
 &= \int_{R} \left| \exp\Big( f_{\theta_1}(x) - f_{\theta_0}(x) \Big) - 1 \right| 
            q(x dx \\
 &= \int_{R} \left( (\theta_1 - \theta_0)^\tran \nabla_{\theta} f_{\bar{\theta}}(x) 
                            \exp\Big( f_{\bar{\theta}}(x) - f_{\theta_0}(x) \Big) \right)^2 q(x) 
             dx 
\end{align*}

First for the upper bound:
\begin{align*}
& \leq \int_{R} \| \theta_1 - \theta_0 \|^2 \| \nabla_{\theta} f_{\bar{\theta}}(x)\|^2
                            \exp\Big( f_{\bar{\theta}}(x) - f_{\theta_0}(x) \Big)
              \exp( f_{\bar{\theta}}(x) ) dx\\
& \leq  \int_{R} \| \theta_1 - \theta_0 \|^2 \| \nabla_{\theta} f_{\bar{\theta}}(x)\|^2
                            \exp\Big( \| \theta_1 - \theta_0\| \|\nabla_{\theta} f_{\tilde{\theta}}(x) \| \Big) 
              \exp( f_{\bar{\theta}}(x) ) dx
\end{align*}

Since we are in $R$, we have that $\| \theta_1 - \theta_0\| \sup_\theta \| \nabla_\theta f_{\theta}(x) \| \leq \log \rho$. We can thus continue the bound:

\begin{align*}
& \leq   \| \theta_1 - \theta_0 \|^2 \int_{R} \| \nabla_{\theta} f_{\bar{\theta}}(x)\|^2
                           e^{\log \rho}
              \exp( f_{\bar{\theta}}(x) ) dx\\
& \leq  \| \theta_1 - \theta_0 \|^2 \rho \int_{-\infty}^\infty \| \nabla_{\theta} f_{\bar{\theta}}(x)\|^2
              \exp( f_{\bar{\theta}}(x) ) dx \\
& \lesssim \| \theta_1 - \theta_0 \|^2
\end{align*}

Now for the lower bound:

\begin{align*}
\alpha^2 &\geq  \int_{R} \left( (\theta_1 - \theta_0)^\tran \nabla_{\theta} f_{\bar{\theta}}(x) \right)^2
                            \exp\Big( - |f_{\bar{\theta}}(x) - f_{\theta_0}(x)| \Big)  \exp(f_{\bar{\theta}}(x))
             dx \\
  &\geq \int_{R} \left( (\theta_1 - \theta_0)^\tran \nabla_{\theta} f_{\bar{\theta}}(x) \right)^2
                            \exp\Big( - \|\theta_1 - \theta_0\| \| \nabla_{\theta} f_{\bar{\theta}}(x) \| \Big)  
          \exp(f_{\bar{\theta}}(x)) dx \\
  &\geq e^{- C_R}  (\theta_1 - \theta_0)^\tran 
                \left( \int_{R} ( \nabla_{\theta} f_{\bar{\theta}}(x) ) 
                                      ( \nabla_{\theta} f_{\bar{\theta}}(x) )^\tran
                          \exp(f_{\bar{\theta}}(x)) dx \right) (\theta_1 - \theta_0)
\end{align*}

Define 
\[
\tilde{G}_{\theta} =  \int_{R} ( \nabla_{\theta} f_{\bar{\theta}}(x) ) 
                                      ( \nabla_{\theta} f_{\bar{\theta}}(x) )^\tran
                          \exp(f_{\bar{\theta}}(x)) dx
\]

For increasing $\rho$ or as $\| \theta_1 - \theta_0 \| \rightarrow 0$, $R \rightarrow \mathbb{R}$, therefore,
$\lambda_{min}(\tilde{G}_{\theta}) \rightarrow \lambda_{min}(G_{\theta}) > 0$. 

Hence, $\alpha^2 \gtrsim \| \theta_1 - \theta_0 \|^2$


\end{proof}




\begin{lemma}
\label{lem:hellinger_theta_equivalence}
Under assumption $D2$, we have that 
\[
\int (\sqrt{p(x)} - \sqrt{q(x)})^2 dx = c \| \theta_0 - \theta_1 \|_2^2
\]
where $ c_{\min} \leq c \leq \frac{1}{4} c_{\max} d_{\Theta} $.
\end{lemma}

\begin{proof}


\begin{align*}
& \int (\sqrt{p(x)} - \sqrt{q(x)})^2 dx \\
& = \int q(x) \left(
        \sqrt{ \frac{p(x)}{q(x)} } - 1 \right)^2 dx \\
&= \int q(x)
       \Big( \exp\Big( f_{\theta_1}(x)/2 - f_{\theta_0}(x)/2 \Big) - 1 \Big)^2 dx \\
\end{align*}

Now, let's look at the exponential term $\exp( f_{\theta_1}(x)/2 - f_{\theta_0}(x)/2 ) - 1$. 

Define $h(\theta) = \exp(f_{\theta}(x)/2 - f_{\theta_0}(x)/2 )$. It is clear that $h(\theta_0) = 1$ and that we wish to bound $h(\theta_1) - h(\theta_0)$. 

\begin{align*}
h(\theta_1) - h(\theta_0) &= (\theta_1 - \theta_0)^\tran \nabla_{\theta} h(\bar{\theta}) \\
   &= \frac{1}{2} (\theta_1 - \theta_0)^\tran \nabla_{\theta} f_{\bar{\theta}}(x) 
             \exp( f_{\bar{\theta}}(x)/2 - f_{\theta_0}(x)/2 )  \\
| h(\theta_1) - h(\theta_0) | &\leq \frac{1}{2} \| \theta_1 - \theta_0 \| 
                  \| \nabla_{\theta} f_{\bar{\theta}}(x) \| \exp( f_{\bar{\theta}}(x)/2 - f_{\theta_0}(x)/2 ) 
\end{align*}

where $\bar{\theta} \in \Theta$ is some convex combination of $\theta_1, \theta_0$.



Thus, we have that
\begin{align*}
\int q(x)
       \Big( \exp\Big( f_{\theta_1}(x)/2 - f_{\theta_0}(x)/2 \Big) - 1 \Big)^2 dx 
  &\leq 
  \int q(x) \frac{1}{4} \| \theta_1 - \theta_0 \|^2 \| \nabla_{\theta} f_{\bar{\theta}}(x) \|^2 
               \exp( f_{\bar{\theta}}(x) - f_{\theta_0}(x)) dx \\
  &\leq \frac{1}{4} \| \theta_1 - \theta_0 \|^2 \int  \| \nabla_{\theta} f_{\bar{\theta}}(x) \|^2  
            \exp( f_{\bar{\theta}} (x)) d x \\
  &\leq \frac{1}{4}  \| \theta_1 - \theta_0 \|^2 \tr( G_{\bar{\theta}} ) \\
  &\leq \frac{1}{4}  \| \theta_1 - \theta_0 \|^2 c_{\max} d_{\Theta}
\end{align*}





\begin{align*}
\int q(x)
       \Big( \exp\Big( f_{\theta_1}(x)/2 - f_{\theta_0}(x)/2 \Big) - 1 \Big)^2 dx 
   &= \int \Big( (\theta_1 - \theta_0)^\tran \nabla_{\theta} f_{\bar{\theta}}(x) \Big)^2 
               \exp(f_{\bar{\theta}}(x))  dx \\
   &= (\theta_1 - \theta_0)^\tran G_{\bar{\theta}} (\theta_1 - \theta_0)  \\
   &\geq c_{\min} \| \theta_1 - \theta_0 \|^2
\end{align*}



\end{proof}





%%% Local Variables:
%%% mode: latex
%%% TeX-master: "../paper"
%%% End:
