
\section{Model and problem formulation}
\label{sec:formulation}

We begin with a formal definition of the weighted SBM and a description of our error metrics for clustering.

\subsection{Model definition}

Consider a network with $n$ nodes and $K \geq 2$ communities. In this paper, we suppose that the communities are approximately balanced; that is, there exists a \emph{cluster-imbalance constant} $\beta$ such that the cluster size $n_k$ for each cluster $k = 1, \dots, K$ satisfies $\frac{\beta n}{K} \geq n_k \geq \frac{n}{\beta K}$. For each node $u$, we let $\sigma(u) \in \{1,2, \dots, K\}$ denote the community assignment of the nodes. 

\begin{definition} (Homogeneous Stochastic Block Model) An edge random variable $A_{uv}$ has the following distribution:
\[
A_{uv} \sim \left\{ \begin{array}{cc}
 Ber(p) & \trm{ if } \sigma(u) = \sigma(v) \\
 Ber(q) & \trm{ if } \sigma(u) \neq \sigma(v) 
\end{array} \right.
\]
\end{definition}

In the more general case of \emph{heterogenous} SBM, we have a $K \times K$ matrix $P$ where each entry $P_{ij} \in [0,1]$. The edge random variable is drawn from $A_{uv} \sim Ber(P_{\sigma(u), \sigma(v)})$. We focus on the homogeneous case in this paper but discuss how to extend our results to the heterogenous setting.

SBM gives a distribution over the set of all networks whose edges are binary. To adapt to networks with continuous edge weights, we generalize the homogenous SBM by adding a second step to the data generating process: an edge weight is sampled from a continuous distribution after it is generated. 

\begin{definition}
\label{def:weighted_homo_sbm1}
(Weighted Homogeneous SBM) Let $0 < P_0, Q_0 < 1$ and let $p(x), q(x)$ be two densities. We first generate the edge presence indicator $Z_{uv}$:
\[
Z_{uv} \sim 
    \left\{ \begin{array}{cc}
    Ber(1-P_0) & \trm{ if } \sigma(u) = \sigma(v) \\
    Ber(1-Q_0) & \trm{ if } \sigma(u) \neq \sigma(v)
   \end{array} \right.
\]
The edge weight random variable is then:

\[
A_{uv} \sim \left\{ \begin{array}{cc} 
     0 & \trm{ if } Z_{uv} = 0 \\
     p(x) & \trm{ if } Z_{uv} = 1 \trm{ and } \sigma(u) = \sigma(v) \\
     q(x) & \trm{ if } Z_{uv} = 1 \trm{ and } \sigma(u) \neq \sigma(v) 
\end{array} \right.
\]

\end{definition}

In this model, an edge is missing with probability either $P_0$ or $Q_0$ depending on whether the potential edge connects two nodes in the same cluster or in different clusters. If the edge is present, then it is given an edge weight drawn from either the density $p(x)$ or $q(x)$, depending again on the nature of the edge. If $p(x)$ and $q(x)$ are Dirac Delta mass at 1, then the weighted homogenous SBM reduces to homogeneous SBM with $p = 1 - P_0$ and $q = 1 - Q_0$. 

\begin{figure}[htp]
\centering
\includegraphics[scale=0.4]{../figs/weightedSBM.jpg}
\caption{weighted stochastic block model}
\label{fig:weighted_stochastic_block_model}
\end{figure}


The model defined in \ref{def:weighted_homo_sbm1} is the focus of our method. However, it is useful to note that we can further generalize model~\ref{def:weighted_homo_sbm1} by allowing both weights and labels. 

\begin{definition} \label{def:weighted_homo_sbm2}
(Weighted and Labeled Homogenous SBM) Let $P, Q$ be two general mixed distributions. The edge random variable $A_{uv}$ is drawn as
\[
A_{uv} \sim \left\{ 
   \begin{array}{cc} 
   P & \trm{ if } \sigma(u) = \sigma(v) \\
   Q & \trm{ if } \sigma(u) \neq \sigma(v)
   \end{array} \right.
\]
\end{definition}
In the case where $P, Q$ are mixed distributions with continuous part $(1-P_0) p(x)$ and $(1-Q_0) q(x)$ respectively and a discrete point mass of $P_0, Q_0$ at zero respectively, then we get back the weighted SBM. 

\subsection{Community estimation}

We study in this paper two problems on the weighted stochastic block model. Our first goal is to find a tractable community recovery algorithm whose misclustering error can be shown to converge to zero at an optimal rate. Our second goal is to find the threshold at which exact recovery of the communities shift from almost never possible to almost always possible. 

\subsubsection{Misclustering error rate}

The goal of a community recovery algorithm is to take as input the adjacency matrix $A$ and try to recover the community assignments. We evaluate a community recovery algorithm by looking at its mis-clustering error rate. To be precise, if $\sigma_0$ is the true clustering and $\hat{\sigma}$ is the clustering generated by a community recovery algorithm, then the misclustering error rate is the following loss function:
\[
l(\hat{\sigma}, \sigma_0) \equiv \min_{\tau \in S_K} \frac{1}{n} \trm{Hamming}(\hat{\sigma},\, \tau \circ \sigma_0 )
\]
where $\trm{Hamming}(\cdot, \, \cdot)$ denotes the Hamming distance. In the definition of mis-clustering error rate, we minimize over the set of permutations $\tau$ on $K$ objects because clusterings are idenfiable only up to a permutation of their labels. It is important to note that $\hat{\sigma}$ is a random quantity both because the community recovery algorithm may be stochastic and because the network $A$ -- the input to the algorithm -- is random. Thus, we aim to bound $l(\hat{\sigma}, \sigma_0)$ in probability. 

Zhang and Zhou~\cite{zhangminimax} and Gao et al~\cite{gao2015achieving} show that the minimax optimal rate of convergence for the unweighted stochastic block model is of the order $\exp\left( - (1 + o(1)) \frac{n I_{\trm{Ber}}}{K} \right)$. $I_{\trm{Ber}} = -2 \log \sqrt{P_0 Q_0} + \sqrt{(1-P_0)(1-Q_0)}$ is the Renyi divergence of order $1/2$ between $\trm{Ber}(P_0)$ and $\trm{Ber}(Q_0)$, where $P_0, Q_0$ are the probabilities of absence for within-community and between-communities edges. Yun and Proutiere have also characterized, though they present the results differently, the optimal rate of convergence for the labeled stochastic block model. Our work extends these results to the weighted SBM and show that the optimal rate is again governed by a Renyi divergence.

Although Renyi divergence is of central importance in homogenous stochastic block model where the cluster sizes are approximately balanced, it is important to note that, in the case of cluster imbalance or in the case of \emph{heterogenous} stochastic block model, Abbe and Sandon \cite{AbbSan15} and Yun and Proutiere \cite{yun2016optimal} have shown that an information divergence that generalizes the Renyi is what drives the intrinsic difficulty of community recovery -- a generalization that is referred to as the CH-divergence.


\subsubsection{Exact recovery}

A closely related problem is that of finding the exact recovery threshold. We say that the weighted stochastic block model has an exact recovery threshold if there is some function of the parameters $\theta(P_0, Q_0, p(x), q(x), K, \beta, n)$ such that exact recovery is asymptotically almost always impossible if $\theta < 1$ and almost always possible if $\theta > 1$. For the homogeneous unweighted stochastic block model, Abbe et al \cite{abbe2014exact} have shown that, when $\beta=1, K=2, 1 - P_0 = \frac{a \log n}{n}$, and $1 - Q_0 = \frac{b \log n}{n}$  (that is, the average degree is of order $\log n$) for some constant $a,b$, then the exact threshold is $\sqrt{a} - \sqrt{b}$ , that is, no exact recovery algorithm can succeed if $\sqrt{a} - \sqrt{b} < 1$ and there exists a recovery algorithm that can succeed with probability tending to one if $\sqrt{a} - \sqrt{b} > 1$.  This result was generalized by Zhang and Zhou \cite{zhangminimax} beyond the $\log n$ degree setting where $ \frac{n I_{\trm{Ber}}}{K \log n}$ was shown to be the threshold. Our paper again extends these results to the weighted stochastic block model where we show that the exact recovery threshold is a natural generalization of the unweighted analogues. 

Apart from exact recovery (also known as strong consistency) and weak recovery, a notion of partial recovery (also known as weak consistency) has also been considered~\cite{MosEtal14, AminiEtal13, zhangminimax}. This notion lies between the other two notions of recovery, and only requires the fraction of misclassified nodes to converge in probability to 0 as $n$ becomes large. A very general result for the $K=2$ case, characterizing when exact and partial recovery are possible for the unweighted homogeneous stochastic block model, is provided in Mossel et al.~\cite{MosEtal14}.



%%% Local Variables:
%%% mode: latex
%%% TeX-master: "paper"
%%% End:
