\message{ !name(paper_all_in_one.tex)}\documentclass{article}
%\documentclass[12pt,pdftex,generic,noinfoline]{imsart}

%TODO
%Modify algorithm to take into account the noise adding state
%Change the tau and other stuff in algorithm descriptions

\RequirePackage[OT1]{fontenc}
\usepackage{amsthm,amsmath,amsfonts,natbib,mathtools,custom_math,amssymb, bbm}
\RequirePackage{hypernat}
\usepackage[ruled,section]{algorithm}
%\usepackage[noend]{algorithmic}
\usepackage{algpseudocode}
\usepackage{graphicx}
%\usepackage[hscale=0.7,vscale=0.8]{geometry}
\usepackage[text={6.0in,8.6in},centering]{geometry}
\usepackage{subfigure}
\usepackage{graphicx}
\usepackage[colorlinks]{hyperref}
\usepackage{yhmath}
\usepackage[usenames,dvipsnames]{xcolor}
\definecolor{shadecolor}{gray}{0.9}
\definecolor{shadecolor}{gray}{0.9}
\hypersetup{citecolor=blue}
\hypersetup{linkcolor=blue}
\hypersetup{urlcolor=blue}
\usepackage{enumerate}
\usepackage{multirow}
\usepackage{verbatim}
\usepackage{color}
%This is for repeating theorems, lemmas
\makeatletter
\newtheorem*{rep@theorem}{\rep@title}
\newcommand{\newreptheorem}[2]{%
\newenvironment{rep#1}[1]{%
 \def\rep@title{#2 \ref{##1}}%
 \begin{rep@theorem}}%
 {\end{rep@theorem}}}
\makeatother
%end repeat theorems
\newreptheorem{theorem}{Theorem} %repeat theorem
\newreptheorem{lemma}{Lemma}      %repeat lemma
\newreptheorem{proposition}{Proposition} %repeat proposition  

\newcommand{\bin}{\text{bin}}
\allowdisplaybreaks

\renewcommand{\baselinestretch}{1.045}
%\parskip12pt
%\parindent0pt
\setcounter{tocdepth}{2}

\input{macros}

\begin{document}

\message{ !name(paper_all_in_one.tex) !offset(581) }



\begin{definition}
For an $n \times n$ matrix $A$ and a permutation $\pi \in S_n$, define $\pi A$ as an $n \times n$ matrix such that $[\pi A]_{uv} = A_{\pi^{-1}(u), \pi^{-1}(v)}$. In other words, $\pi A$ is the result of applying $\pi$ to the rows and columns of $A$. \\

\noindent Let $\hat{\sigma}$ be a clustering algorithm, that is, $\hat{\sigma}(A)$ is a clustering $[n] \rightarrow [K]$ for any $n \times n$ matrix $A$. We say that $\hat{\sigma}$ is \textbf{permutation-invariant} if, for any $A$ and any $\pi \in S_n$, there exists $\tau \in S_K$ such that $$ \hat{\sigma}(\pi A) = \tau \circ \hat{\sigma}(A) \circ \pi^{-1}.$$
\end{definition}

In other words, if $\hat{\sigma}$ is permutation-invariant, then 
$$l(\hat{\sigma}(A) \circ \pi^{-1}, \hat{\sigma}(\pi A)) = 
     \min_{\tau \in S_K} d_H(\tau \circ \hat{\sigma}(A) \circ \pi^{-1}, \hat{\sigma}(\pi A)) = 0$$ 
for any $A$ and any $\pi$. Intuitively, $\hat{\sigma}$ is permutation invariant if the output clustering doesn't dependent upon how the nodes are labeled. To see this, note that $\pi A$ is the result of relabeling the nodes in $A$ by $\pi$ -- $A_{uv} = [\pi A]_{\pi(u), \pi(v)}$. If $\hat{\sigma}$ is permutation-invariant, then the clustering $\hat{\sigma}(\pi A)$ is the result of relabeling the nodes in $\hat{\sigma}(A)$ by $\pi$ -- $\hat{\sigma}(A)(v) = \hat{\sigma}(\pi A)(\pi(v))$.



\message{ !name(paper_all_in_one.tex) !offset(4169) }

\end{document}

%%% Local Variables:
%%% mode: latex
%%% TeX-master: t
%%% End:
