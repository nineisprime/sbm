\message{ !name(paper_all_in_one.tex)}\documentclass{article}
%\documentclass[12pt,pdftex,generic,noinfoline]{imsart}

%TODO
%Modify algorithm to take into account the noise adding state
%Change the tau and other stuff in algorithm descriptions

\RequirePackage[OT1]{fontenc}
\usepackage{amsthm,amsmath,amsfonts,natbib,mathtools,custom_math,amssymb}
\RequirePackage{hypernat}
\usepackage[ruled,section]{algorithm}
%\usepackage[noend]{algorithmic}
\usepackage{algpseudocode}
\usepackage{graphicx}
%\usepackage[hscale=0.7,vscale=0.8]{geometry}
\usepackage[text={6.0in,8.6in},centering]{geometry}
\usepackage{subfigure}
\usepackage{graphicx}
\usepackage[colorlinks]{hyperref}
\usepackage{yhmath}
\usepackage[usenames,dvipsnames]{xcolor}
\definecolor{shadecolor}{gray}{0.9}
\definecolor{shadecolor}{gray}{0.9}
\hypersetup{citecolor=blue}
\hypersetup{linkcolor=blue}
\hypersetup{urlcolor=blue}
\usepackage{enumerate}
\usepackage{multirow}
\usepackage{verbatim}
\usepackage{color}
%This is for repeating theorems, lemmas
\makeatletter
\newtheorem*{rep@theorem}{\rep@title}
\newcommand{\newreptheorem}[2]{%
\newenvironment{rep#1}[1]{%
 \def\rep@title{#2 \ref{##1}}%
 \begin{rep@theorem}}%
 {\end{rep@theorem}}}
\makeatother
%end repeat theorems
\newreptheorem{theorem}{Theorem} %repeat theorem
\newreptheorem{lemma}{Lemma}      %repeat lemma
\newreptheorem{proposition}{Proposition} %repeat proposition  

\newcommand{\bin}{\text{bin}}
\allowdisplaybreaks

\renewcommand{\baselinestretch}{1.045}
%\parskip12pt
%\parindent0pt
\setcounter{tocdepth}{2}

\input{macros}

\begin{document}

\message{ !name(paper_all_in_one.tex) !offset(542) }


In this section, we give a lower bound on the performance of any clustering algorithms on the weighted stochastic block model. For technical reasons, we require the likelihood ratio $\frac{p(x)}{q(x)}$ to be bounded instead of approximately bounded as in Assumption A2 or A2'; we conjecture that the bounded likelihood ratio condition can be relaxed but leave its verification to future works. We also take the true clustering $\sigma_0$ to be random in the sense that $\sigma_0 = \sigma'_0 \circ \pi$ where $\sigma'_0 : [n] \rightarrow [K]$ is any fixed clustering and $\pi \in S_n$ is a random permutation on $[n]$. We take $\sigma_0$ to be random so that a clustering algorithm cannot use any prior information on $\sigma_0$; if $\sigma_0$ were fixed, then the algorithm that trivially outputs $\sigma_0$ would have error 0 for example.

\message{ !name(paper_all_in_one.tex) !offset(3456) }

\end{document}

%%% Local Variables:
%%% mode: latex
%%% TeX-master: t
%%% End:
